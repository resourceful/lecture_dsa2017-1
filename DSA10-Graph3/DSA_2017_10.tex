% Author: Seongjin Lee 
% Gyeongsang National University, Korea 
% 
% 2021-03-03
%

\documentclass[newPxFont,sthlmFooter,nooffset]{beamer}
\usepackage{kotex}
%\usetheme{sthlm}
\usepackage{../style/beamerthemesthlm}
\hypersetup{pdfauthor={Seongjin Lee (insight@gnu.ac.kr)},
            pdfsubject={Data Structure and Algorithm, Lecture Note},
            pdfkeywords={Data Structure, Algorithm, Lecture, Note},
            pdfmoddate={D: \pdfdate},
            pdfcreator={Seongjin Lee}}

%\setbeamertemplate{footline}[text line]{%
%    \parbox{\linewidth}{\vspace*{-8pt} \insertsectionhead  \hfill\insertshortauthor\hfill\insertpagenumber}}
%\setbeamertemplate{navigation symbols}{}

\usepackage{tkz-graph}

\setbeamertemplate{blocks}[rounded]

\title{Data Structure and Algorithm}
\subtitle{Class 10}
\author[SJL]{Seongjin Lee}
\institute{\href{mailto:insight@gnu.ac.kr}{insight@gnu.ac.kr}\\\url{http://resourceful.github.io}\\Systems Research Lab.\\GNU}
\date{2017-03-06} 

\begin{document}



\frame[plain,t]{\titlepage} 

\frame[t]{\frametitle{Table of contents}\tableofcontents} 


%---------------------------------------------------------
%add page -4p KSS-
\section{Review: Kruskal's Algorithm}
\begin{frame}[t, fragile]
	\frametitle{Kruskal's algorithm}
	Kruskal's algorithm builds a minimum cost spanning tree T by adding edges to T one at a time.
	\begin{itemize}
		\item The algorithm selects the edges for inclusion in T in nondecreasing order of their cost.
		\item  An edge is added to T if it does not form a cycle with the edges that are already in T.
		\item  Since G is connected and has n>0 vertices, exactly n-1 edges will be selected for inclusion in T.
	\end{itemize} 
\end{frame}

\section{The code: Kruskal's Algorithm}
\begin{frame}[t, fragile]
  \frametitle{Kruskal's algorithm}
  \lstset{basicstyle=\normalsize, keywordstyle=\color{blue}, language=C}
  \lstinputlisting[firstnumber=1, firstline=1, lastline=8]{codes/kruskal.c}
\end{frame}

\begin{frame}[t, fragile]
  \frametitle{Kruskal's algorithm}
  \lstset{basicstyle=\normalsize, keywordstyle=\color{blue}, language=C}
  \lstinputlisting[firstnumber=10, firstline=10, lastline=18]{codes/kruskal.c}
\end{frame}


\begin{frame}[t, fragile]
  \frametitle{Kruskal's algorithm cntd}
  \lstset{basicstyle=\normalsize, keywordstyle=\color{blue}, language=C}
  \lstinputlisting[firstnumber=19, firstline=19, lastline=28]{codes/kruskal.c}
\end{frame}

\begin{frame}[t, fragile]
  \frametitle{Kruskal's algorithm cntd}
  \lstset{basicstyle=\normalsize, keywordstyle=\color{blue}, language=C}
  \lstinputlisting[firstnumber=30, firstline=30, lastline=40]{codes/kruskal.c}
\end{frame}

\begin{frame}[t, fragile]
  \frametitle{Kruskal's algorithm cntd}
  \lstset{basicstyle=\normalsize, keywordstyle=\color{blue}, language=C}
  \lstinputlisting[firstnumber=42, firstline=42, lastline=53]{codes/kruskal.c}
\end{frame}

\begin{frame}[t, fragile]
  \frametitle{Kruskal's algorithm}
  \lstset{basicstyle=\normalsize, keywordstyle=\color{blue}, language=C}
  \lstinputlisting[firstnumber=55, firstline=55, lastline=58]{codes/kruskal.c}
\end{frame}


\begin{frame}[t, fragile]
  \frametitle{Kruskal's algorithm cntd}
  \lstset{basicstyle=\normalsize, keywordstyle=\color{blue}, language=C}
  \lstinputlisting[firstnumber=75, firstline=75, lastline=86]{codes/kruskal.c}
\end{frame}

\begin{frame}[t, fragile, allowframebreaks]
  \frametitle{Kruskal's algorithm cntd}
  \lstset{basicstyle=\normalsize, keywordstyle=\color{blue}, language=C}
  \lstinputlisting[firstnumber=88, firstline=88, lastline=110]{codes/kruskal.c}
\end{frame}


\begin{frame}[t, fragile]
  \frametitle{Kruskal's algorithm cntd}
  \lstset{basicstyle=\normalsize, keywordstyle=\color{blue}, language=C}
  \lstinputlisting[firstnumber=112, firstline=112, lastline=120]{codes/kruskal.c}
\end{frame}

\begin{frame}[t, fragile]
  \frametitle{Kruskal's algorithm cntd}
  \lstset{basicstyle=\footnotesize, keywordstyle=\color{blue}, language=C}
  \lstinputlisting[firstnumber=122, firstline=122, lastline=138]{codes/kruskal.c}
\end{frame}

\begin{frame}[t, fragile]
  \frametitle{Kruskal's algorithm cntd}
  \lstset{basicstyle=\normalsize, keywordstyle=\color{blue}, language=C}
  \lstinputlisting[firstnumber=140, firstline=140, lastline=150]{codes/kruskal.c}
\end{frame}

\begin{frame}[t, fragile]
  \frametitle{Kruskal's algorithm cntd}
  \lstset{basicstyle=\normalsize, keywordstyle=\color{blue}, language=C}
  \lstinputlisting[firstnumber=151, firstline=151, lastline=164]{codes/kruskal.c}
\end{frame}


\end{document}
