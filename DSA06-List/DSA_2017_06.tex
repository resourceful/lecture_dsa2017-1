% Author: Seongjin Lee 
% Gyeongsang National University, Korea 
% 
% 2021-02-20
%F

\documentclass[newPxFont,sthlmFooter,nooffset]{beamer}
\usepackage{kotex}
%\usetheme{sthlm}
\usepackage{../style/beamerthemesthlm}
\hypersetup{pdfauthor={Seongjin Lee (insight@gnu.ac.kr)},
            pdfsubject={Data Structure and Algorithm, Lecture Note},
            pdfkeywords={Data Structure, Algorithm, Lecture, Note},
            pdfmoddate={D: \pdfdate},
            pdfcreator={Seongjin Lee}}

%\setbeamertemplate{footline}[text line]{%
%    \parbox{\linewidth}{\vspace*{-8pt} \insertsectionhead  \hfill\insertshortauthor\hfill\insertpagenumber}}
%\setbeamertemplate{navigation symbols}{}


\setbeamertemplate{blocks}[rounded]

\title{Data Structure and Algorithm}
\subtitle{Class 6}
\author[SJL]{Seongjin Lee}
\institute{\href{mailto:insight@gnu.ac.kr}{insight@gnu.ac.kr}\\\url{http://resourceful.github.io}\\Systems Research Lab.\\GNU}
\date{2021-02-20} 

\begin{document}



\frame[plain,t]{\titlepage} 

\frame{\frametitle{Table of contents}\tableofcontents} 


%---------------------------------------------------------
\begin{comment}
\section{List} 
\begin{frame}[t]
  \frametitle{Array}
  \begin{itemize}
  \item sequential representation
  \item some operation can be very time- consuming (data movement)
  \item size of data must be predefined
  \item static storage allocation and deallocation
  \end{itemize}

Solution to Data Movement in Sequential Representation
\begin{itemize}
\item<2-> pointers: often called links
\end{itemize}

\end{frame}

\begin{frame}[t]
  \frametitle{Pointers}
for any type T in C
\begin{itemize}
\item there is corresponding type pointer-to-T
\end{itemize}

actual value of pointer type
\begin{itemize}
\item an address of memory
\end{itemize}

pointer operators in C
\begin{itemize}
\item the address operator: \&
\item the dereferencing(or indirection) operator: *
\end{itemize}

\end{frame}

\begin{frame}[t, fragile]
  \frametitle{Pointers}
dynamically allocated storage
\begin{itemize}
\item C provides a mechanism, called a heap, for allocating storage
  at run-time
\end{itemize}

\begin{ncodedef}
int i,*pi;
float f,*pf;
pi = (int *)malloc(sizeof(int));
pf = (float *)malloc(sizeof(float));
*pi = 1024;
*pf = 3.14;
printf(“an integer=%d,a float=%f\n”,*pi,*pf); free(pi);
free(pf);
\end{ncodedef}
\end{frame}
\end{comment}
\section{Singly Linked Lists}
\begin{frame}[t]
  \frametitle{Singly Linked Lists}
  \begin{itemize}
  \item Compose of data part and link part
  \item Link part contains address of the
next element in a list
  \item Non-sequential representations
  \item Size of the list is not predefined
  \item Dynamic storage allocation and deallocation
  \end{itemize}

\includegraphics[width=0.9\textwidth]{figures/fig01_single.png}
\end{frame}
%add image, page composition revised -5p -
\begin{frame}[t]
	\frametitle{Singly Linked Lists: Insert}
	To insert the word \texttt{mat} between \texttt{cat} and \texttt{sat}
	\begin{itemize}
		\item Goal
	\end{itemize}
	\includegraphics[width=0.9\textwidth]{figures/fig01_single.png}
	\begin{center}
		$\big\Downarrow$
	\end{center}
	\includegraphics[width=0.9\textwidth]{figures/fig03_single_delete.png}
\end{frame}

\begin{frame}[t]
  \frametitle{Singly Linked Lists: Insert}
  To insert the word \texttt{mat} between \texttt{cat} and \texttt{sat}
\includegraphics<1>[width=0.9\textwidth]{figures/fig02_single_insert.png}

\includegraphics<2>[width=0.9\textwidth]{figures/fig02_single_insert1.png}

\includegraphics<3>[width=0.9\textwidth]{figures/fig02_single_insert2.png}

\includegraphics<4>[width=0.9\textwidth]{figures/fig02_single_insert3.png}


\begin{enumerate}
\item<1-> Get a node currently unused (\texttt{paddr})
\item<2-> Set the data of this node to mat
\item<3-> Set \texttt{paddr}'s link to point to the address found in the link of
  the node \texttt{cat}
\item<4-> Set the link of the node cat to point to \texttt{paddr}

\end{enumerate}
\end{frame}

%add image, page composition revised -7p KSS-
\begin{frame}[t]
	\frametitle{Singly Linked Lists: Delete}
	To delete \texttt{mat} from the list
	\begin{itemize}
		\item Goal
	\end{itemize}
	\includegraphics[width=0.9\textwidth]{figures/fig03_single_delete.png}
	\begin{itemize}
		\item Result
	\end{itemize}
	\includegraphics[width=0.9\textwidth]{figures/fig01_single.png}
\end{frame}

\begin{frame}[t]
  \frametitle{Singly Linked Lists: Delete}
To delete \texttt{mat} from the list
\includegraphics<1>[width=0.9\textwidth]{figures/fig03_single_delete.png}

\includegraphics<2>[width=0.9\textwidth]{figures/fig03_single_delete1.png}

\includegraphics<3>[width=0.9\textwidth]{figures/fig03_single_delete2.png}
\begin{enumerate}
\item<1-> Find the element that immediately precedes \texttt{mat}, which is cat
\item<2-> Set its link to point to \texttt{mat}'s link
\item<3-> Free the \texttt{mat}'s node
\end{enumerate}
\end{frame}


\begin{frame}[t]
  \frametitle{Singly Linked List}
\bigskip
  \begin{block}{Note}
    \begin{center}
\bigskip

      There is no data movement in insert and delete operation


\bigskip
    \end{center}

  \end{block}
%add time complexity -9p KSS-
Time Complexity
\begin{itemize}
	\item O(1)
\end{itemize}
\end{frame}

\begin{frame}[t]
  \frametitle{Singly Linked List: Features}
  
Required capabilities to make linked representations possible
begin{itemize}
\begin{itemize}
\item A mechanism for defining a node’s structure, that is, the
  field it contains
\item A way to create new nodes when we need them
\item A way to remove nodes that we no longer need
\end{itemize}

\end{frame}

\begin{frame}[t, fragile]
  \frametitle{Singly Linked List: The data type}
  Define the node structure for the list
  \begin{itemize}
  \item \textbf{data field}: Character array
  \item \textbf{link field}: Pointer to the next node
    \begin{itemize}
    \item Self-referential structure
    \end{itemize}
  \end{itemize}

  \begin{ncodedef}
typedef struct node {
    char data[4];
    struct node *next; // self-referential structure
} node_t;
  \end{ncodedef}
  \begin{center}
    \includegraphics[width=4cm]{figures/fig04_node.png}
  \end{center}

\end{frame}

\begin{frame}[t, fragile]
  \frametitle{Singly Linked List: Creating a node}
Create new nodes for our list then place the word \texttt{bat} into the
list
\begin{columns}
  \begin{column}{0.5\textwidth}
    \begin{ncodedef}
node_t *head = NULL;
head = malloc(sizeof(node_t));
if (head == NULL) {
    return 1;
}
strcpy(head->data, "bat");
head->next = NULL;
\end{ncodedef}

  \end{column}
  \begin{column}{0.5\textwidth}
    \includegraphics<1>[width=6cm]{figures/fig05_head.png}
    \includegraphics<2>[width=6cm]{figures/fig05_head1.png}
  \end{column}
\end{columns}
\end{frame}

\begin{frame}[t, fragile]
  \frametitle{Singly Linked List: Adding a node}
\begin{ncodedef}
node_t second; 
    
second = malloc(sizeof(node_t)); 
  
second->next=NULL;
strcpy(second->data, "cat");
  
head->next=second;
\end{ncodedef}
    \includegraphics<1>[width=0.9\textwidth]{figures/fig05_second1.png}
    \includegraphics<2>[width=0.9\textwidth]{figures/fig05_second2.png}
\end{frame}

\begin{frame}[t, fragile]
  \frametitle{Singly Linked List: Implementation}
  \begin{itemize}
  \item Data structure
  \item Insert a node
  \item Delete the list
  \item Remove a node
  \item Search
  \item Print
  \end{itemize}

  \begin{ncodedef}
typedef struct node {
    int data;
    struct node *next; 
} node_t;
  \end{ncodedef}  

\end{frame}


\begin{frame}[t, fragile]
  \frametitle{Singly Linked List: Print}
\begin{ncodedef}
void print_list(node_t * head) {
    node_t * current = head;

    while (current != NULL) {
        printf("%d\n", current->data);
        current = current->next;
    }
}  
\end{ncodedef}
\bigskip
  \begin{itemize}
  \item Create a pointer that iterates the list
  \item Print the data in the list.
  \item Repeat until the pointer reaches the end of the list
    (the next node in \texttt{NULL})
  \end{itemize}

\end{frame}

\begin{frame}[t, fragile]
  \frametitle{Singly Linked List: Insert at the end}

  \begin{ncodedef}
void insert_tail(node_t * head, int data) {
    node_t * current = head;
    while (current->next != NULL) {
        current = current->next;
    }

    /* now we can add a new variable */
    current->next = malloc(sizeof(node_t));
    current->next->data = data;
    current->next->next = NULL;
}
  \end{ncodedef}  
\bigskip
  \begin{itemize}
  \item Iterate the list using a pointer to find the end of the list
  \item Add a new node at the end of the list
  \item Add the data
  \item Mark next node as \texttt{NULL}
  \end{itemize}

\end{frame}

\begin{frame}[t, fragile]
  \frametitle{Singly Linked List: Insert at the beginning}

\begin{ncodedef}
void insert_head(node_t ** head, int data) {
    node_t * new_node;
    new_node = malloc(sizeof(node_t));

    new_node->data = data;
    new_node->next = *head;
    *head = new_node;
}  
\end{ncodedef}
\bigskip
\begin{itemize}
\item Create a new node with data
\item Link the new node to point to the head of the list
\item Set the head of the list with the new node
\end{itemize}

\end{frame}


\begin{frame}[t]
  \frametitle{Singly Linked List: Insert at the beginning}
    \includegraphics[width=0.9\textwidth]{figures/fig06_pointer.png}  

\bigskip
    \includegraphics[width=0.9\textwidth]{figures/fig06_pointer1.png}  

\bigskip
    \includegraphics[width=0.9\textwidth]{figures/fig06_pointer2.png}  
\end{frame}


\begin{frame}[t, fragile]
  \frametitle{Singly Linked List: Removing the first node}

\begin{ncodedef}
int remove_head(node_t ** head) {
    int retval = -1;
    node_t * new_head = NULL;
    if (*head == NULL) {
        return -1;
    }
    new_head = (*head)->next;
    retval = (*head)->data;
    free(*head);
    *head = new_head;
    return retval;
}  
\end{ncodedef}
\begin{itemize}
\item Select the next item that head points
\item Save it as a new head
\item Free the head
\item Use the new head as the head of the list
\end{itemize}
\end{frame}


\begin{frame}[t, fragile]
  \frametitle{Singly Linked List: Removing the last node}
\begin{ncodedef}
int remove_tail(node_t * head) {
    int retval = 0;
    if (head->next == NULL) { // only one node in the list
        retval = head->data;
        free(head);           // remove the head
        return retval;
    }

    node_t * current = head;
    while (current->next->next != NULL) { // go to the second last node
        current = current->next;
    }

    retval = current->next->data; // get the last node's data
    free(current->next);         // free the last node
    current->next = NULL;        // set the next node NULL
    return retval;

}  
\end{ncodedef}
\end{frame}


\begin{frame}[t, fragile]
  \frametitle{Singly Linked List: Removing a value }
  \begin{ncodedef}
int remove_value(node_t ** head, int data) {
    node_t *previous, *current;
    if (*head == NULL)
        return -1;

    if ((*head)->data == data) 
        return pop(head);

    previous = current = (*head)->next;
    while (current != NULL) {
        if (current->data == data) {
            previous->next = current->next;
            free(current);
            return data;
        }

        previous = current;
        current  = current->next;
    }
    return -1;
}    
  \end{ncodedef}
  
\end{frame}


\begin{frame}[t, fragile]
  \frametitle{Singly Linked List: Deleting the list }
  \begin{ncodedef}
void delete_list(node_t *head) {
    node_t  *current = head, 
            *next = head;

    while (current) {
        next = current->next;
        free(current);
        current = next;
    }
}    
  \end{ncodedef}
\end{frame}

\section{Dynamically Linked Stacks and Queues}
\begin{frame}[t]
  \frametitle{Representing Stack and Queue by Linked List}
  \begin{figure}
    \centering
    \includegraphics[width=0.9\textwidth]{figures/fig07_stack_list.png}
     \caption{linked list stack}
\end{figure}
\bigskip
\begin{figure}
  \centering
  \includegraphics[width=0.9\textwidth]{figures/fig07_queue_list.png}
  \caption{linked list queue}
\end{figure}
\end{frame}


\begin{frame}[t, fragile]
  \frametitle{Representing stacks by linked list}
  \begin{columns}
    \begin{column}{0.5\textwidth}
  \begin{ncodedef}
/* n=MAX_STACKS=10 */ 
#define MAX_STACKS 10

typedef struct {
    int key; 
    /* other fields */ 
} element;

typedef struct stack *stack_ptr; 

typedef struct stack {
    element item;
    stack_ptr link; 
};

stack_ptr top[MAX_STACKS];    
  \end{ncodedef}      
    \end{column}
    \begin{column}{0.5\textwidth}
      \begin{itemize}
      \item Initial condition for stacks
        \begin{itemize}
        \item \texttt{top[i] = NULL, \\0 $\leq$ i $\leq$ MAX\_STACKS}
        \end{itemize}
      \item Boundary condition for \texttt{n} stacks
        \begin{itemize}
        \item Empty condition
          \begin{itemize}
          \item \texttt{top[i] = NULL} \\iff the $i^{th}$ stack is empty
          \end{itemize}

        \item Full condition
          \begin{itemize}
          \item \texttt{IS\_FULL(temp)} \\iff the memory is full
          \end{itemize}

        \end{itemize}

      \end{itemize}
    \end{column}
  \end{columns}

\end{frame}


\begin{frame}[t, fragile]
  \frametitle{Pushing to a linked list stack}
  \begin{ncodedef}
void push(stack_ptr *ptop, element item) { 
    stack_ptr temp = (stack_ptr)malloc(sizeof (stack)); 
    if(IS_FULL(temp)) {
        fprintf(stderr,``The memory is full\n'');
        exit(1); 
    }
    temp->item=item; 
    temp->link=*ptop; 
    *ptop = temp;
}    
  \end{ncodedef}
\end{frame}


\begin{frame}[t, fragile]
  \frametitle{Poping a linked list stack}
  \begin{ncodedef}
element pop(stack_ptr *ptop) { 
    stack_ptr temp = *ptop; 
    element item; 
    if(IS_EMPTY(temp)) {
        fprintf(stderr,``The stack is empty\n'');
        exit(1); 
     }
     item=temp->item; 
     *ptop=temp->link; 
     free(temp); 
     return item;
}    
  \end{ncodedef}
\end{frame}

\begin{frame}[t, fragile]
  \frametitle{Representing Queues by linked list}
  \begin{columns}
    \begin{column}{0.5\textwidth}
  \begin{ncodedef}
/* m=MAX_QUEUES=10 */ 
#define MAX_QUEUES 10 

typedef struct queue *queue_ptr;

typedef struct queue {
    element item;
    queue_ptr link; 
};

queue_ptr front[MAX_QUEUES], 
          rear[MAX_QUEUES];    
  \end{ncodedef}      
    \end{column}
    \begin{column}{0.5\textwidth}
      \begin{itemize}
      \item Initial condition for queue
        \begin{itemize}
        \item \texttt{front[i] = NULL, \\0 $\leq$ i $\leq$ MAX\_QUEUES}
        \end{itemize}
      \item Boundary condition for queues
        \begin{itemize}
        \item Empty condition
          \begin{itemize}
          \item \texttt{front[i] = NULL} \\iff the $i^{th}$ queue is empty
          \end{itemize}

        \item full condition
          \begin{itemize}
          \item \texttt{IS\_FULL(temp)} \\iff the memory is full
          \end{itemize}

        \end{itemize}


      \end{itemize}
    \end{column}
  \end{columns}

\end{frame}

\begin{frame}[t, fragile]
  \frametitle{Add to the rear of a linked list queue}
  \begin{ncodedef}
void insert(queue_ptr *pfront, 
            queue_ptr *prear, 
            element item) {
    queue_ptr temp = (queue_ptr)malloc(sizeof(queue));

    if(IS_FULL(temp)) {
        fprintf(stderr, ``The memory is full\n''); 
        exit(1);
    } 

    temp->item=item; 
    temp->link=NULL; 
    
    if (*pfront) 
        (*prear)->link=temp; 
    else
        *pfront = temp; 

    *prear = temp;
}    
  \end{ncodedef}
\end{frame}

\begin{frame}[t, fragile]
  \frametitle{Delete from the front of a linked list queue}
  \begin{ncodedef}
element delete(queue_ptr *pfront) { 
    queue_ptr temp=*pfront;
    element item;

    if (IS_EMPTY(*pfront)) {
        fprintf(stderr,``The queue is empty\n'');
        exit(1); 
    }

    item=temp->item; 
    *pfront=temp->link; 
    free(temp);

    return item;
}    
  \end{ncodedef}
\end{frame}

\begin{frame}[t]
  \frametitle{Representing Stack and Queues by Linked List}
Advantages are
\begin{itemize}
\item No data movement is necessary: O(1) operation
\item No full condition check is necessary
\item Size is growing and shrinking dynamically
\end{itemize}


\end{frame}


\section{Doubly Linked Lists}
\begin{frame}[t]
  \frametitle{Doubly Linked Lists}
Problems of singly linked lists
\begin{itemize}
\item Move to only one way direction
\item Hard to find the previous node
\item Hard to delete the arbitrary node
\end{itemize}


\bigskip
Doubly linked circular list
\begin{itemize}
\item Doubly lists + circular lists
\item Allow two links
\item Two way direction
\end{itemize}


\end{frame}

\begin{frame}[t, fragile]
  \frametitle{Doubly Linked Lists}
  \begin{ncodedef}
typedef struct node *node_ptr; 

typedef struct node {
    node_ptr llink; 
    element item; 
    node_ptr rlink;
};    
  \end{ncodedef}

  \begin{center}
    \includegraphics<2>[width=0.9\textwidth]{figures/fig08_double.png}
  \end{center}


\end{frame}

\begin{frame}[t, fragile]
  \frametitle{Doubly Linked Lists}
Suppose that \texttt{ptr} points to any node in a doubly linked list

\begin{ncodedef}
ptr 

ptr->llink->rlink

ptr->rlink->llink
\end{ncodedef}


\end{frame}
%frametitle revised -38,39p KSS-
\begin{frame}[t]
  \frametitle{Doubly Linked Circular Lists}
Introduce dummy node, called, head
\begin{itemize}
\item To represent empty list
\item Make easy to implement operations 
\item Contains no information in item field
\end{itemize}
\bigskip
\begin{figure}
  \begin{center}
    \includegraphics[width=0.4\textwidth]{figures/fig09_double_head.png}
    \caption{empty doubly linked circular list with head node}
  \end{center}
\end{figure}
\end{frame}

\begin{frame}[t]
  \frametitle{Doubly Linked Circular Lists}
  \begin{figure}
  \begin{center}
    \includegraphics[width=0.8\textwidth]{figures/fig09_double_head1.png}
    \caption{doubly linked circular list with head node}
  \end{center}
\end{figure}
\end{frame}

\begin{frame}[t, fragile]
  \frametitle{Insert into a doubly linked circular list}
  \begin{ncodedef}
void d_insert(node_ptr node, node_ptr newnode) {
    /* insert newnode to the right of node */ 
    newnode->llink = node;
    newnode->rlink = node->rlink; 

    node->rlink->llink = newnode; 
    node->rlink = newnode;
}    
  \end{ncodedef}

Time complexity O(1)
\end{frame}

\begin{frame}[t]
  \frametitle{Insert into a doubly linked circular list}

Insertion into an empty doubly linked circular list

    \includegraphics[width=0.4\textwidth]{figures/fig10_double_insert.png}
    \includegraphics<2>[width=0.4\textwidth]{figures/fig10_double_insert1.png}
  
\end{frame}

\begin{frame}[t, fragile]
  \frametitle{Deletion from a doubly linked circular list}
  \begin{ncodedef}
void ddelete(node_ptr node, node_ptr deleted) {
    /* delete from the doubly linked list */ 
    if (node == deleted)
        printf("Deletion of head node not permitted.\n");
    else {
        deleted->llink->rlink = deleted->rlink; 
        deleted->rlink->llink = deleted->llink; 
        free(deleted);
    } 
}    
  \end{ncodedef}

Time complexity: O(1)
\end{frame}

\begin{frame}[t]
  \frametitle{Deletion from a doubly linked circular list}
Deletion in a doubly linked circular list

    \includegraphics[width=0.4\textwidth]{figures/fig11_double_delete.png}
    \includegraphics<2>[width=0.4\textwidth]{figures/fig11_double_delete1.png}
  
\end{frame}

\begin{frame}[t]
  \frametitle{Notes on doubly linked circular list}
  \begin{itemize}
  \item Don't have to traverse a list - time complexity of O(1)
  \item Insert(delete) front/middle/rear is all the same
  \end{itemize}
\end{frame}

%add HW -44p KSS-
\begin{frame}[t]
\frametitle{Homework : Music Player}
Function
\begin{itemize}
	\item Play : Print current data
	\item Next : Move next music
	\item Back : Move back music
	\item Add Music : Add new music
	\item Delete Music : Delete current music
	\item Circular : If use the Next function with last music, move first music
\end{itemize}
\begin{flushright}
\textbf{Reference answer with cont'd skeleton code}
\end{flushright}
\end{frame}

\begin{frame}[t, fragile, allowframebreaks]
	\frametitle{Homework : Music Player}
	\begin{lstlisting}
	
	typedef struct music 
	{
	char musicName[15]; 
	struct music *backMusic;
	struct music *frontMusic;
	} Music;
	
	Music *headerPointer;
	Music *currentMusic;
	
	int main(void)
	{
	int i;
	
	initMusic();
	currentMusic = headerPointer;
	char command[10];
	
	//Code is required.
	
	allFree();
	return 0;
	}
	
	void initMusic()
	{
	headerPointer = (Music *)malloc(sizeof(Music));
	headerPointer -> backMusic = headerPointer;
	headerPointer -> frontMusic = headerPointer;
	}
	
	void playMusic()
	{
	//Code is required.
	}
	
	void Next()
	{
	//Code is required.
	}
	
	
	void Back()
	{
	//Code is required.
	}
	
	
	void addMusic()
	{
	char newdata[15];
	scanf("%s", newdata);
	Music *newMusic;
	newMusic = (Music *)malloc(sizeof(Music));
	
	//Code is required.
	}
	
	
	void deleteMusic()
	{
	if(headerPointer -> frontMusic == headerPointer)
	{
	printf("There is no music here\n");
	}
	else
	{
	Music *deletePointer;
	deletePointer = currentMusic;
	
	//Code is required.
	
	free(deletePointer);
	}
	}
	
	void allFree()
	{
	Music *iter = headerPointer;
	Music *iterNext = NULL;
	do {
	iterNext = iter -> frontMusic;
	free(iter);
	iter = iterNext;
	} while(iter != headerPointer);
	
	printf("____________________\n\n");
	}
	\end{lstlisting}
\end{frame}

\end{document}


\end{document}
