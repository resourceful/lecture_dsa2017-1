% Author: Seongjin Lee 
% Gyeongsang National University, Korea 
% 
% 2021-02-07
%

\documentclass[newPxFont,sthlmFooter,nooffset]{beamer}
\usepackage{kotex}
%\usetheme{sthlm}
\usepackage{../style/beamerthemesthlm}
\hypersetup{pdfauthor={Seongjin Lee (insight@gnu.ac.kr)},
            pdfsubject={Data Structure and Algorithm, Lecture Note},
            pdfkeywords={Data Structure, Algorithm, Lecture, Note},
            pdfmoddate={D: \pdfdate},
            pdfcreator={Seongjin Lee}}

%\setbeamertemplate{footline}[text line]{%
%    \parbox{\linewidth}{\vspace*{-8pt} \insertsectionhead  \hfill\insertshortauthor\hfill\insertpagenumber}}
%\setbeamertemplate{navigation symbols}{}


\setbeamertemplate{blocks}[rounded]

\title{Data Structure and Algorithm}
\subtitle{Class 3}
\author[SJL]{Seongjin Lee\\
\footnotesize{Updated by Songsub Kim}}
\institute{\href{mailto:insight@gnu.ac.kr}{insight@gnu.ac.kr}\\\url{http://resourceful.github.io}\\Systems Research Lab.\\GNU}
%date update -2021.02.02 kimsongsub-
\date{} %2021-02-07

\begin{document}



\frame[plain,t]{\titlepage} 

\frame{\frametitle{Table of contents}\tableofcontents} 


%---------------------------------------------------------
\section{Array} 
\begin{frame}[t, allowframebreaks, fragile]
  \frametitle{Array}
an \textbf{array} is a set of pairs, \textbf{<index, value>}, such that each index that is defined has a value associated with it
\begin{itemize}
\item  ``a consecutive set of memory locations'' in C
\item  logical order is the same as physical order
\end{itemize}

operations
\begin{itemize}
\item creating a new array
\item retrieves a value
\item stores a value
\item insert a value into array - delete a value at the array
\end{itemize}
\framebreak

ADT: Array

\begin{itemize}
  \item \textbf{Object}: A set of pairs \textit{<index, value>} where for each value of \textit{index} ther is a value from the set \textit{item}. \textit{Index} is a finite ordered set of one or more dimensions.
  \item \textbf{Functions}: for all $a \in Array$, $i \in index$, $x \in item$, $j$, $size \in $ integer
  \begin{itemize}
    \item \textit{Array} \texttt{create(j, list)}: \textbf{return} an array of $j$ dimension
    \item \textit{Item} \texttt{Retrieve(A, i)}: \textbf{if} ($i \in index$) \textbf{return} the item in index value $i$ in array $A$ \textbf{else return} error 
    \item \textit{Array} \texttt{Store(A, i, x)}: \textbf{if} ($i$ in $index$) \textbf{return} an array that is identical to array $A$ excep the new pair $<i, x>$ has been inserted \textbf{else return} error 
  \end{itemize}
\end{itemize}
\framebreak

An one-dimensional array in C is declared implicitly by appending brackets to the name of a variable

\begin{codedef}
  int list[5], int *plist[5];
\end{codedef}

Always remember that starting index of array is \texttt{0} in C  

\framebreak
Let's consider implementing an one-dimensional arrays

\begin{codedefnb}
int list[5];
\end{codedefnb}

\begin{itemize}
\item allocates five consecutive memory locations
\item each memory location is large enough to hold a single integer
\item base address is address of the first element
\end{itemize}

\begin{codedefnb}
list = &list[0]
\end{codedefnb}
%add table -2021.02.03 kimsongsub-
\begin{center}
	\begin{tabular}{| c | c | c | c | c |}
		\hline
		\texttt{list[0]} & \texttt{list[1]} & \texttt{list[2]} & \texttt{list[3]} & \texttt{list[4]} \\
		\hline
		 trash value & trash value & trash value & trash value & trash value \\
		\hline
	\end{tabular}
\end{center}

\framebreak
  
  \begin{tabular}{l  l}
    Variable & Memory Address \\ \hline
\texttt{\&list[0]} & base address = $\alpha$ \\
\texttt{\&list[1]} & $\alpha +$ \texttt{sizeof(int)}\\
\texttt{\&list[2]} & $\alpha + 2 \cdot$ \texttt{sizeof(int)}\\
\texttt{\&list[3]} & $\alpha + 3 \cdot$ \texttt{sizeof(int)}\\
\texttt{\&list[4]} & $\alpha + 4 \cdot$ \texttt{sizeof(int)}\\
  \end{tabular}
%add additional explanation -2021.02.06 kimsongsub-
\begin{itemize}
	\item Because different architectures have different int sizes, we have to use ``sizeof''
\end{itemize}

\bigskip
\texttt{\&list[i]} in a C programs
\begin{itemize}
\item C interprets it as a pointer to an integer or its value
\end{itemize}

\framebreak

\begin{codedef}
int *list1;          // pointer variable to an int
\end{codedef}

\begin{codedef}
int *list2[5];       // list2 : pointer constant to an int, 
                     // and five memory locations for holding 
                     // integers are reserved
\end{codedef}
%add example code -2021.02.06 kimsongsub-
\begin{codedef}
int list[] = { 0,1,2,3,4,5 };
	
printf("%d\n", list[3]);    // 3
printf("%d\n", *(&list[0]+3));    // 3
printf("%d\n", *(list+3));    // 3 ...All is the same expression
\end{codedef}
\texttt{(list2+i)} equals \texttt{\&list2[i]}, and \texttt{*(list2+i)} equals \texttt{list2[i]}
\begin{itemize}
\item regardless of the type of the array list2
\end{itemize}

\framebreak
%add Page, example -2021.02.07 kimsongsub-

How about this ?
\begin{codedef}
printf("%d\n", ++list[0]);
\end{codedef}

\begin{itemize}
	\item \texttt{++list[0] -> list[0]+1}
	\item the result is same as \texttt{list[1]}
\end{itemize}

\framebreak

consider the way C treats an array when it is a parameter to a function
\begin{itemize}
\item the parameter passing is done using call-by-value in C
\item but array parameters have their values altered
\end{itemize}

\end{frame}

\begin{frame}[t, fragile]
  \frametitle{Practice Problem (5 min)}
  Array: Ex. 2.1 Analyze and comprehend the code before running it
%typo, indent revised -2021.02.03 kimsongsub-
\begin{codedef}
#define MAX_SIZE 100
float sum(float [], int); 
float input[MAX_SIZE], answer; 
int i;
void main(void) {
    for(i = 0; i < MAX_SIZE; i++) 
    	input[i] = i;
    answer = sum(input, MAX_SIZE);
    printf{"The sum is: %f\n", answer);
}     

float sum(float list[], int n) {
    int i;
    float tempsum = 0;
    for(i = 0; i < n; i++)
        tempsum += list[i];
    return tempsum;
}
\end{codedef}
\end{frame}



\begin{frame}[t, allowframebreaks, fragile]
  \frametitle{Recap: On Pointer}
Pointer Variable stores address
\begin{itemize}
\item \& : Starting address of allocated variable
\item * : Value stored on the address of the pointer variable
\end{itemize}
\lstinputlisting[lineskip=1pt, numbers=left]{codes/pointer_variable.c}


\framebreak
%typo, revised -2021.02.03 kimsongsub-
  Do Not!
  \begin{itemize}
  \item pointer variable is not referencing an address, so cannot store a value
\begin{codedef}
int *ptr;
*ptr = 100;
\end{codedef}
%add addional explanation -2021.02.07 kimsongsub-
It is better to handle NULL for the pointer that do not refer to address right away
\end{itemize}

\framebreak


Do Not!
\begin{itemize}
\item the data type must equal
\begin{codedef}
double Pi = 3.14;
int *pPi = &Pi; 
\end{codedef}
\item cannot dereference a non-pointer variable
\begin{codedef}
int num;
*num = 100;  
\end{codedef}
\end{itemize}

\framebreak


Do!
  \begin{itemize}
\item it is recommended to initialize a pointer value with NULL ('\textbackslash \texttt{0}') \\ 
See the example
  \end{itemize}
\lstinputlisting[lineskip=1pt, numbers=left]{codes/null_pointer.c}

\framebreak


\begin{codedef}
#include <stdlib.h>

void *malloc(size_t size); // allocates size bytes of memory and returns a pointer to the allocated memory

void *free(void *ptr); // frees allocation that were created via the preceding allocation function

void *calloc(size_t count, size_t size); // contiguously allocates enough space for count objects that are size bytes of memory each and returns a pointer to the allocated memory. The allocated memory is filled with bytes of value zero.

void *realloc(void *ptr, size_t size); // change the size of the allocation pointed to by ptr to size, and returns ptr
\end{codedef}


\framebreak
%add free explanation page -2021.02.07 kimsongsub-

if the memory is not freed after being allocated, a memory leak will occur

\begin{figure}[h]
	\centering
	\includegraphics[width=0.33\textwidth]{figures/fig03_memory.png}
\end{figure}
\end{frame}


\begin{frame}[t, fragile]
  \frametitle{Practice Problem (5 Min)}
  Array: Ex 2.2, 1-dimensional array addressing
\begin{codedef}
int one[] = {0, 1, 2, 3, 4};   
\end{codedef}

write a function that prints out both the address of the $i^{th}$ element of the array and the value found at this address

\end{frame}

\begin{frame}[t, allowframebreaks, fragile]
  \frametitle{Practice Problem (Solution)}
  Array: Ex 2.2, 1-dimensional array addressing

  \lstinputlisting[lineskip=-1pt]{codes/array_address.c}

One-dimensional array accessed by address
\begin{itemize}
\item address of $i^{th}$ element \texttt{ptr + i}
\item obtain the value of the $i^{th}$ value \texttt{*(ptr + i)}
\end{itemize}

\framebreak


\begin{codedefnb}
Address		Contents
1518325632	        0
1518325633	        1
1518325634	        2
1518325635	        3
1518325636	        4
\end{codedefnb}
\bigskip
One-dimensional array addressing
\begin{itemize}
\item the addresses increase by two on an Intel 386 machine
\item Example shown is the result of Mac OS X on Intel Core i5 Machine
\end{itemize}

\end{frame}


\section{Structures and Unions}

\begin{frame}[t, allowframebreaks, fragile]
  \frametitle{Structures and Unions: struct}
\textbf{struct}  
\begin{itemize}
\item structure or record
\item the mechanism of grouping data
\item permits the data to vary in type
\end{itemize}
\bigskip

collection of data items where
\begin{itemize}
\item each item is identified as to its type and name
\end{itemize}


\framebreak


\begin{codedef}
struct {
    char name[10];
    int age;
    float salary;
} person;
\end{codedef}
\bigskip
creating a variable
\begin{itemize}
\item whose name is person and
\item has three fields
  \begin{enumerate}
  \item a name that is a character array
  \item an integer value representing the age of the person
  \item a float value representing the salary of the individual
  \end{enumerate}

\end{itemize}


\framebreak


% add additional descript -2021.02.05 kimsongsub-
use of the .(period) as the structure member operator

\begin{codedef}
strcpy(person.name, "james");
person.age = 30;
person.salaray = 35000;  
\end{codedef}
\bigskip
\begin{itemize}
\item select a particular member of the structure
\end{itemize}

 

\framebreak


\texttt{typedef statement}
\begin{itemize}
\item create our own structure data type
\end{itemize}

type 1
\begin{codedef}
typedef struct human_being { 
    char name[10];
    int age;
    float salary;
} human;
\end{codedef}

type 2
\begin{codedef}
typedef struct { 
    char name[10]; 
    int age;
    float salary;
} human_being;
\end{codedef}

\framebreak


\texttt{human\_being}  
\begin{itemize}
\item the name of the type defined by the structure definition
\end{itemize}

\begin{codedef}
human_being person1, person2;

if(strcmp(person1.name, person2.name)) 
    printf("The two people do not have the same name\n");
else
    printf("The two people have the same name\n");
\end{codedef}
\end{frame}


\begin{frame}[t, fragile]
  \frametitle{Structures and Unions: Assignment}
\textbf{assignment}
\begin{itemize}
\item permits structure assignment in ANSI C

\begin{codedef}
  person1 = person2;
\end{codedef}
\item but, in most earlier versions of C assignment of structures is not permitted
\begin{codedef}
   strcpy(person1.name,person2.name); 
   person1.age=person2.age; 
   person1.salary=person2.salary;    
\end{codedef}
\end{itemize}
\end{frame}


\begin{frame}[t, fragile]
  \frametitle{Structures and Unions: Equality or Inequality}
\textbf{equality or inequality}
\begin{itemize}
\item cannot be directly checked
\item Example function to check equality of struct
\begin{codedef}
int humans_equal(human_being person1, human_being person2) { 
    if(strcmp(person1.name,person2.name))
        return FALSE; 
    if(person1.age!=person2.age)
        return FALSE; 
    if(person1.salary!=person2.salary)
        return FALSE; 
    return TRUE;
}    
\end{codedef}
\end{itemize}

\end{frame}


\begin{frame}[t, allowframebreaks, fragile]
  \frametitle{Structures and Unions: Embedding Structure}
Embedding of a structure within a structure
\begin{codedef}
typedef struct { 
    int month; 
    int day;
    int year;
} date;

typedef struct human_being { 
    char name[10];
    int age;
    float salary;
    date dob; // embedded structure
};  
\end{codedef}

\framebreak


Ex. A person born on Feb 14 1992
% add declare person1 -2021.02.05 kimsongsub-
\begin{codedef}
human_being person1;
person1.dob.month = 2;
person1.dob.day = 14;
person1.dob.year = 1992;
\end{codedef}
\end{frame}


\begin{frame}[t, allowframebreaks, fragile]
  \frametitle{Structures and Unions: Unions}
\textbf{Unions}
\begin{itemize}
\item similar to a structure, but
\item the fields of a union must share their memory space
\item only one field of the union is ``active'' at any given time
\end{itemize}

\framebreak


\begin{codedef}
typedef struct sex_type {
    enum tag_field {female,male} sex; 
    union {
        int children;
        int beard; } u;
};

typedef struct human_being {
    char name[10]; 
    int age;
    float salary; 
    date dob; 
    sex_type sex_info;
};

human_being person1,person2;    
\end{codedef}

\framebreak


Assign values to \texttt{person1} and \texttt{person2}
\begin{codedef}
person1.sex_info.sex = male; 
person1.sex_info.u.beard = FALSE; /* FALSE: 0 */
\end{codedef}

and

\begin{codedef}
person2.sex_info.sex = female; 
person2.sex_info.u.children = 4;
\end{codedef}
\end{frame}

% add structures and unions quiz page-2021.02.07 kimsongsub-
\begin{frame}[t, allowframebreaks, fragile]
	\frametitle{Differences between structure and union}
What are differences between structure and union ? 
\begin{itemize}
	\item Structure
\end{itemize}
\begin{codedef}
typedef struct example{
	int x, y;
	double d;
}struct_example;

struct_example se;

printf("%d\n", sizeof(se));
printf("%p %p %p\n", &se.x, &se.y, &se.d);
\end{codedef}

>> 16
\newline>> 0x7ff0 0x7ff4 0x7ff8
\begin{itemize}
	\item 16byte size, different address
\end{itemize}

\framebreak



\begin{itemize}
	\item Union
\end{itemize}
\begin{codedef}
	typedef union example{
		int x, y;
		double d;
	}union_example;

	union_example ue;
	
	printf("%d\n", sizeof(ue));
	printf("%p %p %p\n", &ue.x, &ue.y, &ue.d);
\end{codedef}

>> 8
\newline>> 0x7ff0 0x7ff0 0x7ff0 
\begin{itemize}
	\item 8byte size, same address
\end{itemize}

\framebreak



\begin{codedef}
struct {
    int i, j; float a, b;
}  
\end{codedef}

or 
\begin{codedef}
struct {
    int i; int j; float a; float b;
};
\end{codedef}

stored in the same way
\begin{itemize}
\item increasing address locations in the order specified in the
  structure definition
\end{itemize}

size of an object of a struct or union type
\begin{itemize}
\item the amount of storage necessary to represent the largest
  component
\end{itemize}

\end{frame}

\begin{frame}[t, allowframebreaks, fragile]
  \frametitle{Structures and Unions: {\large Self-referential structures}}
  \begin{itemize}
  \item one or more of its components is a pointer to itself
  \item usually require dynamic storage management routines to
    explicitly obtain and release memory
  \end{itemize}
\begin{codedef}
typedef struct list {
    char data;
    list *link;
};    
\end{codedef}
the value of \texttt{link}
\begin{itemize}
\item address in memory of an instance of \texttt{list} or \texttt{null} pointer
\end{itemize}

\framebreak



\begin{codedef}
list item1, item2, item3;
item1.data = 'a';
item2.data = 'b';
item3.data = 'c';
item1.link = item2.link = item3.link = NULL;   
\end{codedef}
\begin{uncoverenv}<2->
attach these structures together
\begin{codedef}
item1.link = &item2;
item2.link = &item3;
\end{codedef}
\end{uncoverenv}
\end{frame}




\section{Sparse Matrices}

\subsection{The Abstract Data Type}
\begin{frame}[t, allowframebreaks, fragile]
	\frametitle{The Abstract Data Type}
Matrix is a mathematical object that is used to solve many problems in the natural sciences
\begin{itemize}
  \item our interest centers not only on the specification of an appropriate ADT
  \item but also in finding representations that let us efficiently perform the operations described in specification
\end{itemize}	

\framebreak
A matrix contains $m$ rows and $n$ columns of elements 

\begin{itemize}
  \item write as $m \times n$ and read as $m$ by $n$ ($m$ rows, $n$ columns)
  \item use two-dimensional array
  \item space complexity \\
        $S(m, n) = m *n$
  \end{itemize}

\begin{equation*}
  \begin{blockarray}{r *{3}{c}}
      & col0 & col1 & col2 \\
\begin{block}{l [*{3}{c}]}
 row0 & -27  &   3  &  4   \\
 row1 &   6  &  82  &  -2  \\
 row2 & 109  & -64  &  11   \\
 row3 &  12  &   8  &  9   \\
 row4 &  48  &  27  & 47   \\
\end{block}
  \end{blockarray}
\end{equation*}

\framebreak

When a matrix is represented as a two-dimensional array defined as \texttt{a[MAX$\_$ROWS][MAX$\_$COLS]}
\begin{itemize}
  \item we can locate quikcly any element by writing \texttt{a[i][j]}
  \item $i$ is the row index, $j$ is the column index
\end{itemize}

\framebreak
There are some problems with \texttt{a[i][j]} notation.
\begin{itemize}
  \item a matrix with many zero's: \textit{sparse matrix}
\end{itemize}

\texttt{A[6,6]}

\begin{equation*}
  \begin{blockarray}{r *{6}{c}}
      & col0 & col1 & col2 & col3 & col4 & col5 \\
\begin{block}{l [*{6}{c}]}
 row0 & 15  & 0   & 0  &  22 &   0 &   15   \\
 row1 & 0   & 11  & 3  &   0 &   0 &   0   \\
 row2 & 0   & 0   & 0  &  -6 &   0 &   0   \\
 row3 & 0   & 0   & 0  &   0 &   0 &   0   \\
 row4 & 91  & 0   & 0  &   0 &   0 &   0   \\
 row5 & 0   & 0   & 28 &   0 &   0 &   0   \\
\end{block}
  \end{blockarray}
\end{equation*}

\framebreak

common characteristics of a sparse matrix
\begin{itemize}
\item most elements are zero's 
\item inefficient memory utilization
\end{itemize}
\bigskip

solutions
\begin{itemize}
\item store only nonzero elements
\item using the triple \texttt{<row,col,value>}
\item must know
  \begin{itemize}
  \item the number of rows
  \item the number of columns
  \item the number of non-zero elements
  \end{itemize}

\end{itemize}
\framebreak
We first must consider the operations that we want to perform on these Matrices
\begin{itemize}
  \item matrix creation
  \item addtion
  \item muliplication
  \item transpose
\end{itemize}

\framebreak
\textbf{ADT}: \textit{Sparse Matrix}
\begin{itemize}
  \item objects: a set of triples $<row, column, value>$, where $row$ and $column$ are integers and form a unique combination, and $value$ comes from the set $item$
  \item Functions: for all $a, b \in SparseMatrix, x \in item, i, j, maxCol, maxRow \in index$
\end{itemize}
\framebreak
\begin{itemize}
  \item \textit{SparseMatrix Create( maxRow, maxCol)}
  \begin{itemize}
    \item Return: a \textit{SparseMatrix} that can hold up to $maxItems = maxRow \times maxCol$ and whose maximum row size is $maxRow$ and whose maximum colum size is $maxCol$
  \end{itemize}
  \item \textit{Sparse Matrix Transpose}
  \begin{itemize}
    \item Return: the matrix produced by intechaging the row and column value of every triple
  \end{itemize}
  \item \textit{SparseMatrix Add(a, b)}
  \begin{itemize}
    \item Return: if the dimensions of $a$ and $b$ are the same \\
    return the matrix produced by adding corresponding items, namely those with identical $row$ and $column$ values \\
    else return error
  \end{itemize}
  \item \textit{SparseMatrix Multiply(a,b)}
  \begin{itemize}
    \item Return: if number of columns in $a$ equals number or rows in $b$ \\
    return the matrix $d$ produced by multiplying $a$ by $b$ according to the formula: $d[i][j] = \sum (a[i][k]\cdot b[k][j])$ where $d(i,j)$ is the $(i,j)$th element \\ 
    else return error 
  \end{itemize}
\end{itemize}




\end{frame}

\subsection{Sparse Matrix Representation}

\begin{frame}[t, allowframebreaks, fragile]
  \frametitle{Sparse Matrix Representation}
Before implementing any of the ADT operations
\begin{itemize}
  \item we must establish the representation of the sparse matrix
  \item We can characterize unquely any element within a matrix by using the triple $<row, col, value>$
\end{itemize}

Other considerations
\begin{itemize}
  \item We want transpose operation to work efficiently, we should organize the triples so that the row indices are in ascending order
  \item Also requiring that all the triples for any row be stored so that the column indices are in ascending order
  \item To ensure that the operations terminate, we must know the number of rows and columns, and the number of nonzero elements in the matrix
\end{itemize}


\framebreak

\textit{SparseMatrix} \texttt{Create(maxRow, maxCol)}:

\begin{codedef}
#define MAX_TERMS 101  /* max number of terms +1 */

typedef struct {
    int col; 
    int row; 
    int value;
} term;

term a[MAX_TERMS];  
\end{codedef}

\begin{itemize}
\item a[0].row: the number of rows
\item a[0].col: the number of columns
\item a[0].value: the total number of non-zeros
\item choose row-major order
\end{itemize}

\framebreak

  \begin{center}
    \begin{tabular}{r c c c}
      & row & col & value \\ \hline
      a[0]&  6  &  6  &   8   \\
      a[1]&  0  &  0  &  15   \\ 
      a[2]&  0  &  3  &  22   \\ 
      a[3]&  0  &  5  & -15   \\ 
      a[4]&  1  &  1  &  11   \\ 
      a[5]&  1  &  2  &   3   \\ 
      a[6]&  2  &  3  &  -6   \\ 
      a[7]&  4  &  0  &  91   \\ 
      a[8]&  5  &  2  &  28   \\ 
    \end{tabular}
  \end{center}
  \begin{itemize}
  \item space complexity (*variable space requirement) \\
\texttt{S(m,n) = 3 * t} where \\
\texttt{t} : the number of non-zero’s
\item independent of the size of rows and columns
  \end{itemize}

\end{frame}

\begin{frame}[t]
	\frametitle{Practice Problem (5 Min)}
Write down how the following matrix is represented in our definition of sparse matrix
\begin{equation*}
  \begin{blockarray}{r *{6}{c}}
      & col0 & col1 & col2 & col3 & col4 & col5 \\
\begin{block}{l [*{6}{c}]}
 row0 & 15  & 0   & 0  &  22 &   0 &   15   \\
 row1 & 0   & 11  & 3  &   0 &   0 &   0   \\
 row2 & 0   & 0   & 0  &  -6 &   0 &   0   \\
 row3 & 0   & 0   & 0  &   0 &   0 &   0   \\
 row4 & 91  & 0   & 0  &   0 &   0 &   0   \\
 row5 & 0   & 0   & 28 &   0 &   0 &   0   \\
\end{block}
  \end{blockarray}
\end{equation*}

\end{frame}


\subsection{Transposing A Matrix}

\begin{frame}[t, allowframebreaks, fragile]
  \frametitle{Transposing A Matrix}
Transposing the sample matrix
\begin{itemize}
\item interchange rows and columns
\item move \texttt{a[i][j]} to \texttt{a[j][i]}
\end{itemize}


\begin{columns}
  \begin{column}{0.5\textwidth}
sample matrix
    \begin{tabular}{r c c c}
      & row & col & value \\ \hline
      a[0]&  6  &  6  &   8   \\
      a[1]&  0  &  0  &  15   \\ 
      a[2]&  0  &  3  &  22   \\ 
      a[3]&  0  &  5  & -15   \\ 
      a[4]&  1  &  1  &  11   \\ 
      a[5]&  1  &  2  &   3   \\ 
      a[6]&  2  &  3  &  -6   \\ 
      a[7]&  4  &  0  &  91   \\ 
      a[8]&  5  &  2  &  28   \\ 
    \end{tabular}
   
  \end{column}
  \begin{column}{0.5\textwidth}
transposed matrix
    \begin{tabular}{r c c c}
      & row & col & value \\ \hline
      b[0]&  6  &  6  &   8   \\
      b[1]&  0  &  0  &  15   \\ 
      b[2]&  0  &  4  &  91   \\ 
      b[3]&  1  &  1  &  11   \\ 
      b[4]&  2  &  1  &   3   \\ 
      b[5]&  2  &  5  &  28   \\ 
      b[6]&  3  &  0  &  22   \\ 
      b[7]&  3  &  2  &  -6   \\ 
      b[8]&  5  &  0  &  -15   \\ 
    \end{tabular}  
  \end{column}
\end{columns}
  
\framebreak

\textbf{Algorithm: BAD\_TRANSPOSE} 
\begin{codedef}
for each row i
    take element <i, j, value>;
    store it as element <j,i,value> of the transpose; 
end;  
\end{codedef}
The problem

We will not know exactly where to place element $<j, i, value>$ in the transpose matrix until we have processed all the elements taht precede

\begin{center}
  \begin{tabular}{l l l}
    (0, 0, 15), & which becomes & (0, 0, 15)\\
    (0, 3, 22), & which becomes & (3, 0, 22)\\
    (0, 5, -15), & which becomes & (5, 0, -15)\\
  \end{tabular}
\end{center}

If we place these triples consecutively in the transpose matrix, then, as we insert new tiples we must move elements to maintain the correct order

\framebreak

We can avoid this data movement by using the column indices to determine the placement of elments in the transpose matrix

\textbf{Algorithm TRANSPOSE}
\begin{codedef}
for all elements in column j
    place element <i,j,value> in element <j,i,value>
end for;
\end{codedef}

Since the original matrix ordered the rows, the columns within each row of the tranpose matrix will be arraged in ascending order as well

\framebreak



\begin{lstlisting}[frame=single, lineskip = -1pt, numbers = left, framexleftmargin=15pt, framexrightmargin=-5pt, xleftmargin = 25pt ]
void transpose(term a[], term b[]) {
    /* b is set to the transpose of a */
    int n, i, j, currentb;
    n = a[0].value;       // total number of elements
    b[0].row = a[0].col;  // rows in b = columns in a
    b[0].col = a[0].row;  // columns in b = rows in a
    b[0].value = n;
    if(n > 0){            // non zero matrix
        currentb = 1;
        for (i = 0; i < a[0].col; i++)
        /* transpose by the columns in a */
            for(j = 1; j <= n; j++)
            /* find elements from the current column */
                if(a[j].col == i) {
                /* element is in current column, add it to b */
                  b[currentb].row = a[j].col;  
                  b[currentb].col = a[j].row;  
                  b[currentb].value = a[j].value;  
                  currentb++;
                }
            }
    }
}
\end{lstlisting}

\framebreak
\textbf{Analysis of transpose}

Computing Time
\begin{itemize}
  \item nested \textbf{for} loops are the decisive factor
  \item two \textbf{if} statements and several assignments statments requires on constant Time
  \item outer \textbf{for} loop is iterated \texttt{a[0].col} times
  \item inner \textbf{for} loop requires \texttt{a[0].value} times
  \item the total time for the nested \textbf{for} loop is $columns\cdot elements$ : $O(columns\cdot elements)$
\end{itemize}

The problem

\begin{itemize}
\item unnecessary loop for each column
\end{itemize}

\framebreak
In essence 
\begin{codedef}
for (j = 0; j < columns; j++)
    for (i = 0; i < rows; i++)
        b[j][i] = a[i][j];
\end{codedef}

The $O(columns \cdot elements)$ time for our transpose function becomes $O(columns^2 \cdot rows)$ 
\begin{itemize}
  \item when the number of elements iof the order $columns \cdot rows$
\end{itemize}

Solution: 
\begin{itemize}
  \item Use a bit more storage 
\end{itemize}
\end{frame}

\begin{frame}[t,fragile]
	\frametitle{Practice Problem (5 Min)}
Find the pros and cons of the transpose algorithm introduced in slide 22

\begin{lstlisting}[frame=single, lineskip = -1pt, numbers = left,  numberstyle={\scriptsize}, framexleftmargin=15pt, framexrightmargin=-5pt, xleftmargin = 25pt ]
void transpose(term a[], term b[]) {
  /* b is set to the transpose of a */
  int n, i, j, currentb;
  n = a[0].value;       // total number of elements
  b[0].row = a[0].col;  // rows in b = columns in a
  b[0].col = a[0].row;  // columns in b = rows in a
  b[0].value = n;
  if(n > 0){            // non zero matrix
      currentb = 1;
      for (i = 0; i < a[0].col; i++)
      /* transpose by the columns in a */
          for(j = 1; j <= n; j++)
          /* find elements from the current column */
              if(a[j].col == i) {
              /* element is in current column, add it to b */
                b[currentb].row = a[j].col;  
                b[currentb].col = a[j].row;  
                b[currentb].value = a[j].value;  
                currentb++;
          }
  }
}
\end{lstlisting}   
	

\end{frame}

\subsection{Fast Transpose Algorithm}
\begin{frame}[t, fragile, allowframebreaks]
  \frametitle{Fast Transpose Algorithm}



Create better algorithm by using a little more storage
\begin{itemize}
\item row\_terms the number of element in each row
\item starting\_pos the starting point of each row
\end{itemize}


We can transpose a matrix represneted as a sequence of triples in $O(columns + elements)$ time
\begin{itemize}
  \item determining the number of elelments in each column of original matrix (number of elements in each row)
  \item we can determine the starting position of each row in the transpose matrix
  \item we can move the elements in the original maxtrix one by one into their correct position
\end{itemize}

\framebreak


\begin{ncodedef}
void fast_transpose(term a[], term b[]){

    /* the transpose of a is placed in b */
    int row_terms[MAX_COL];    // number of rows in column
    int starting_pos[MAX_COL]; // column counts
    
    int i, j;

    // origianl matrix 
    int numCols = a[0].col;    // number of columns
    int numTerms = a[0].value; // number of elements

    // transposed matrix
    b[0].row = numCols;        // number of rows       
    b[0].col = a[0].row;        // number of columns
    b[0].value = numTerms;     // number of elements
    
    // for non zero matrix
    if(numTerms > 0) {       
        // initializing matrix
        for(i = 0; i < numCols; i++) 
            row_terms[i] = 0;

        // number of elements in a row
        for(i = 1; i <= numTerms; i++)
            row_terms[a[i].col]++;

        starting_pos[0] = 1;

        // accounting column position 
        for(i = 1; i < numCols; i++)
            starting_pos[i] = starting_pos[i-1] + row_terms[i-1];

        // transposing
        for(i = 1; i <= numTerms; i++){
            j = starting_pos[a[i].col]++;
            b[j].row = a[i].col;
            b[j].col = a[i].row;
            b[j].value = a[i].value;
        }
    }
}  
\end{ncodedef}

\framebreak

Intial values: 
\begin{codedefnb}
numCols = a[0].col = 6
numTerms = a[0].value = 8

b[0].row = numCols = 6
b[0].col = a[0].row = 6
b[0].value = numTerms = 8
\end{codedefnb}

{\scriptsize
\begin{tabular}{r c c c}
  & row & col & value \\ \hline
  a[0]&  6  &  6  &   8   \\
  a[1]&  0  &  0  &  15   \\ 
  a[2]&  0  &  3  &  22   \\ 
  a[3]&  0  &  5  & -15   \\ 
  a[4]&  1  &  1  &  11   \\ 
  a[5]&  1  &  2  &   3   \\ 
  a[6]&  2  &  3  &  -6   \\ 
  a[7]&  4  &  0  &  91   \\ 
  a[8]&  5  &  2  &  28   \\ 
\end{tabular}
}

\framebreak
Number of elements in a row 

\begin{columns}
  \begin{column}{0.5\textwidth}
    \begin{lstlisting}[frame=single, lineskip = -1pt, numbers = left, firstnumber=20, framexleftmargin=15pt, framexrightmargin=-5pt, xleftmargin = 25pt ]
// initializing matrix
for(i = 0; i < numcols; i++) 
 row_terms[i] = 0;  

// number of elements in a row
for(i = 1; i <= numTerms; i++)
 row_terms[a[i].col]++;  
\end{lstlisting}
\end{column}
  \begin{column}{0.5\textwidth}
{\scriptsize
    \begin{tabular} {r | c |c |c}
      code & line 22 & \texttt{a[i].col} & line 26 \\ \hline
      \texttt{row\_terms[0]} & 0  & 1+1 & 2 \\
      \texttt{row\_terms[1]} & 0  & 1   & 1 \\
      \texttt{row\_terms[2]} & 0  & 1+1 & 2 \\
      \texttt{row\_terms[3]} & 0  & 1+1 & 2 \\
      \texttt{row\_terms[4]} & 0  & -   & 0 \\
      \texttt{row\_terms[5]} & 0  & 1   & 1 \\
    \end{tabular}
}        
  \end{column}

\end{columns}



{\scriptsize
\begin{tabular} {c | c | c}
   & & \texttt{row\_terms} \\ 
  i &\texttt{a[i].col}  & \texttt{[a[i].col]++} \\ \hline
  1 & 0 & 1 \\
  2 & 3 & 1 \\
  3 & 5 & 1 \\
  4 & 1 & 1 \\
  5 & 2 & 1 \\
  6 & 3 & 2 \\
  7 & 0 & 2 \\
  8 & 2 & 2 \\
\end{tabular}
}
\framebreak

\texttt{starting\_pos} caculation

\begin{columns}
  \begin{column}{0.5\textwidth}
\begin{lstlisting}[frame=single, lineskip = -1pt, numbers = left, firstnumber=28, framexleftmargin=15pt, framexrightmargin=-5pt, xleftmargin = 25pt ]
starting_pos[0] = 1;

// accounting column position 
for(i = 1; i < numcols; i++)
  starting_pos[i] = 
  starting_pos[i-1]+row_terms[i-1];
\end{lstlisting}      
  \end{column}
  \begin{column}{0.5\textwidth}
    {\scriptsize
\begin{tabular} {l | c | c}
                       &  \texttt{start\_pos[i-1]} & \\
\texttt{start\_pos[i]} &     \texttt{+rowterm[i-1]} & result \\ \hline
\texttt{start\_pos[0]} & 1 & 1\\
\texttt{start\_pos[1]} & 1+2 & 3\\
\texttt{start\_pos[2]} & 3+1& 4\\
\texttt{start\_pos[3]} & 4+2& 6\\
\texttt{start\_pos[4]} & 6+2& 8\\
\texttt{start\_pos[5]} & 8+0& 8\\
\end{tabular}
    }    
  \end{column}
\end{columns}


Transposing routine

\begin{columns}
  \begin{column}{0.45\textwidth}
\begin{lstlisting}[frame=single, lineskip = -1pt, numbers = left, firstnumber=34, framexleftmargin=15pt, framexrightmargin=-5pt, xleftmargin = 25pt ]
// transposing
for(i = 1; i <= numTerms; i++){
  j = starting_pos[a[i].col]++;
  b[j].row = a[i].col;
  b[j].col = a[i].row;
  b[j].value = a[i].value;
\end{lstlisting}      
  \end{column}
  \begin{column}{0.55\textwidth}
    {\scriptsize
\begin{tabular} {c| c | c |  c | c }
  & & \texttt{start\_pos} & & \\
i & \texttt{a[i].col}& \texttt{[a[i].col]++} & a[i] & b[j] \\ \hline
1 & 0 & 1 & \texttt{a[1]} & \texttt{b[1]} \\
2 & 3 & 6 & \texttt{a[2]} & \texttt{b[6]} \\
3 & 5 & 8 & \texttt{a[3]} & \texttt{b[8]} \\
4 & 1 & 3 & \texttt{a[4]} & \texttt{b[3]} \\
5 & 2 & 4 & \texttt{a[5]} & \texttt{b[4]} \\
6 & 3 & 7 & \texttt{a[6]} & \texttt{b[7]} \\
7 & 0 & 2 & \texttt{a[7]} & \texttt{b[2]} \\
8 & 2 & 5 & \texttt{a[8]} & \texttt{b[5]} \\
\end{tabular}
    }    
  \end{column}
\end{columns}


\framebreak
final form 
{\scriptsize
\begin{tabular}{r |c c c || c | c c c || c | c c c}
  ori & row & col & value  & trans & row & col & value & final & row & col & value \\ \hline
  a[0]&	6	&	6	&	8	&	\texttt{b[0]}	&	6	&	6	&	8	&	\texttt{b[0]}	&	6	&	6	&	8	\\
  a[1]&	0	&	0	&	15	&	\texttt{b[1]}	&	0	&	0	&	15	&	\texttt{b[1]}	&	0	&	0	&	15	\\
  a[2]&	0	&	3	&	22	&	\texttt{b[6]}	&	0	&	3	&	22	&	\texttt{b[2]}	&	0	&	4	&	91	\\
  a[3]&	0	&	5	&	-15	&	\texttt{b[8]}	&	0	&	5	&	-15	&	\texttt{b[3]}	&	1	&	1	&	11	\\
  a[4]&	1	&	1	&	11	&	\texttt{b[3]}	&	1	&	1	&	11	&	\texttt{b[4]}	&	2	&	1	&	3	\\
  a[5]&	1	&	2	&	3	&	\texttt{b[4]}	&	1	&	2	&	3	&	\texttt{b[5]}	&	2	&	5	&	28	\\
  a[6]&	2	&	3	&	-6	&	\texttt{b[7]}	&	2	&	3	&	-6	&	\texttt{b[6]}	&	3	&	0	&	22	\\
  a[7]&	4	&	0	&	91	&	\texttt{b[2]}	&	4	&	0	&	91	&	\texttt{b[7]}	&	3	&	2	&	-6	\\
  a[8]&	5	&	2	&	28	&	\texttt{b[5]}	&	5	&	2	&	28	&	\texttt{b[8]}	&	5	&	0	&	-15	\\
\end{tabular}
}

\framebreak
\textbf{Analysis of \texttt{fast\_transpose()}}
\begin{itemize}
  \item First two \texttt{for} loops compute the values for $rowTerms$
  \begin{itemize}
    \item comptuting time: $numCols$ and $numTerms$
  \end{itemize}
  \item the thrid \texttt{for} loop carries out the computation of $startingPos$
  \begin{itemize}
    \item comptuting time: $numCols-1$
  \end{itemize}
  \item the last \texttt{for} loop places the triples into the transpose matrix
  \begin{itemize}
    \item comptuting time: $numTerms$
  \end{itemize}
  \item Complexity of the algorithm : $O(columns+elements)$
  \begin{itemize}
    \item Worst case : $O(columns \cdot elements)$ when number of elements is of the order $columns \cdot elements$
  \end{itemize}
\end{itemize}


\end{frame}

\begin{frame}[t,fragile]
	\frametitle{Practice Problem (5 Min)}
Analyze and understand the algorithm of \texttt{fast\_transpose}

\begin{lstlisting}[frame=single, lineskip = -1pt, numbers = none, numberstyle={\scriptsize}, framexleftmargin=15pt, framexrightmargin=-25pt, xleftmargin = 25pt ]
void fast_transpose(term a[], term b[]){
  ...
    if(numTerms > 0) {       
        for(i = 0; i < numcols; i++) 
            row_terms[i] = 0;   
        for(i = 1; i <= numTerms; i++)
            row_terms[a[i].col]++;
        starting_pos[0] = 1;
        for(i = 1; i < numcols; i++) 
            starting_pos[i] = starting_pos[i-1] + row_terms[i-1];
        for(i = 1; i <= numTerms; i++){ 
            j = starting_pos[a[i].col]++;
            b[j].row = a[i].col;
            b[j].col = a[i].row;
            b[j].value = a[i].value;
        }
    }
}  
\end{lstlisting}


\end{frame}



\begin{comment}
\section{Representation of Multidimensional Arrays}



\begin{frame}[t, allowframebreaks, fragile]
  \frametitle{Representation of Multidimensional Arrays}
internal representation of multidimensional arrays
\begin{itemize}
\item how to state n-dimensional array into 1-dimensional array ?
\item how to retrieve arbitrary element in an array
\texttt{  a[upper$^0$][upper$^1$] $\cdots$ [upper$^{n-1}$]}
\end{itemize}

The number of elements in the array
$\prod ^{n-1}_{i=0} upper^{i}$

Ex. a[10][10][10]
\begin{itemize}
\item $10\times 10 \times 10 = 1000$ (units)
\end{itemize}

\framebreak



represent multidimensional array by
\begin{itemize}
\item in what order ?
\end{itemize}


\textbf{row-major-order}
\begin{itemize}
\item store multidimensional array by rows
\end{itemize}
\includegraphics[width=0.7\textwidth]{figures/fig01_multi_dim.png}

\framebreak



how to retreive values
\begin{itemize}
\item starting-address + offset-value
\item assume α: starting-address
\end{itemize}

\textbf{1-dimensional} array \texttt{a[u$^0$]}

\begin{center}
  \begin{tabular}{r r} \hline
    \&a[0] & : $\alpha$ \\
    \&a[1] & : $\alpha + 1$ \\[-0.5em]
    $\cdot$ & $\cdot$ \\[-0.5em]
    $\cdot$ & $\cdot$ \\[-0.5em]
    $\cdot$ & $\cdot$ \\
    \&a[u$^0$-1] & : $\alpha + (u^0 -1)$ \\ \hline
    \&a[i] & : $\alpha + i$ \\
  \end{tabular}
\end{center}

\framebreak



\textbf{2-dimensional} array \texttt{a[u$^0$][u$^1$]}

\includegraphics[width=0.7\textwidth]{figures/fig02_2-dim_array.png}

\texttt{\&a[i][j] =} $\alpha+i\cdot u^1 + j$

\framebreak



\textbf{2-dimensional} array \texttt{a[u$^0$][u$^1$][u$^2$]}
\begin{eqnarray*}
\&a[i][j][k] &=& \alpha + i\cdot u^1 \cdot u^2 + j\cdot u^2 +k \\
             &=& \alpha + u^2[i \cdot u^1 + j] +k
\end{eqnarray*}

\texttt{general} case array \texttt{a[u$^0$][u$^1$] $\cdots$[u$^{n-1}$}]

$\&a[i_0][i_1][i_{n-1}]$
\begin{equation*}
  = \alpha + \sum^{n-1}_{j=0} i_j \cdot a_j
  \begin{cases}
    a_j = \prod^{n-1}_{k=j+1} u_k & \text{for} ~j < n-1\\
    a_{n-1} = 1 & \text{for} ~j = n-1
  \end{cases}
\end{equation*}
\end{frame}
\end{comment}









\end{document}
