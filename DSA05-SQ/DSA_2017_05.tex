% Author: Seongjin Lee 
% Gyeongsang National University, Korea 
% 
% 2017-03-06
%

\documentclass[newPxFont,sthlmFooter,nooffset]{beamer}
\usepackage{kotex}
%\usetheme{sthlm}
\usepackage{../style/beamerthemesthlm}
\hypersetup{pdfauthor={Seongjin Lee (insight@gnu.ac.kr)},
            pdfsubject={Data Structure and Algorithm, Lecture Note},
            pdfkeywords={Data Structure, Algorithm, Lecture, Note},
            pdfmoddate={D: \pdfdate},
            pdfcreator={Seongjin Lee}}

%\setbeamertemplate{footline}[text line]{%
%    \parbox{\linewidth}{\vspace*{-8pt} \insertsectionhead  \hfill\insertshortauthor\hfill\insertpagenumber}}
%\setbeamertemplate{navigation symbols}{}


\setbeamertemplate{blocks}[rounded]

\title{Data Structure and Algorithm}
\subtitle{Class 5}
\author[SJL]{Seongjin Lee}
\institute{\href{mailto:insight@gnu.ac.kr}{insight@gnu.ac.kr}\\\url{http://resourceful.github.io}\\Systems Research Lab.\\GNU}
\date{2017-03-06} 

\begin{document}



\frame[plain,t]{\titlepage} 

\frame{\frametitle{Table of contents}\tableofcontents} 


%---------------------------------------------------------
\section{Stack} 
\begin{frame}[t]
  \frametitle{Stack Abstract Data Type}
\textbf{Stack and Queue} is
\begin{itemize}
\item special cases of the more general data type, \textbf{Ordered List}
\end{itemize}

\textbf{ADT Stack}
\begin{itemize}
\item Ordered List
\item Insertions and deletions are made at one end called the top
\end{itemize}
\end{frame}

\begin{frame}[t]
  \frametitle{Stack Abstract Data Type}
Given stack $S = ( a_0, \ldots, a_{n-1})$
\begin{itemize}
\item $a_0$ : Bottom element
\item $a_{n-1}$ : Top element
\item $a_{i}$ : On top of element $a_{i-1}$ $(0 < i < n)$
\end{itemize}

A.K.A Last-In-First-out (LIFO)

\bigskip

Inserting and deleting elements in a stack
{\footnotesize  \begin{centering}
\begin{tabular}{!{\onslide<2->}c<{\onslide<3->}c<{\onslide<4->}c<{\onslide<5->}c<{\onslide<6->}c<{\onslide<7->}c}
push A & $\rightarrow$ push B &  $\rightarrow$ push C &  $\rightarrow$ push D &  $\rightarrow$ push E &  $\rightarrow$ pop E \\
    \end{tabular}
  \end{centering}}

{\onslide<1-> stack state}

{\footnotesize  \begin{centering}
\begin{tabular}{!{\onslide<2->} c<{\onslide<2->}c<{\onslide<3->}c<{\onslide<3->}c<{\onslide<4->}c<{\onslide<4->}c<{\onslide<5->}c<{\onslide<5->}c<{\onslide<6->}c<{\onslide<6->}c<{\onslide<7->}c<{\onslide<7->}c}
      & &   & &    &  &   &  & E & $\leftarrow$ top &   & \\
      & &   &  &   &  & D & $\leftarrow$ top & D  & & D &  $\leftarrow$ top\\
      & &   &  & C & $\leftarrow$ top & C  & & C & & C &\\
      & & B & $\leftarrow$ top & B & & B & & B & & B &\\
      A & $\leftarrow$ top & A & & A & & A & & A & & A &\\
    \end{tabular}
  \end{centering}}
\end{frame}

\begin{frame}[t]
  \frametitle{Stack Abstract Data Type: System Stack}
Stack is used by a program at run-time to process function calls

Activation record (stack frame) initially contains only 
\begin{itemize}
\item a pointer to to the previous stack frame
\item a return address
\end{itemize}
If this invokes another function
\begin{itemize}
\item local variables
\item parameters of the invoking function
\end{itemize}

\end{frame}


\begin{frame}[t]
  \frametitle{Stack Abstract Data Type: System Stack}
System Stack after function call

{\onslide<2-> Run-time program simply creates a new stack frame
  \begin{itemize}
  \item also for each recursive call
  \end{itemize}
}

  \begin{center}
    \begin{tabular}{!{\onslide<1->}c<{\onslide<2->}c}
      \includegraphics[width=0.4\textwidth]{figures/fig03_stack1.png} &
      \includegraphics[width=0.4\textwidth]{figures/fig03_stack2.png} \\
    \end{tabular}
  \end{center}

\end{frame}


\begin{frame}[t,fragile]
  \frametitle{Stack Abstract Data Type}
\begin{codedef}
~\textbf{Structure:}~    Stack is 
   ~\textbf{Objects:}~ a finite ordered list with zero or more elements
   ~\textbf{Functions:}~
   For all stack ~$\in$~ Stack, 
   item ~$\in$~ element,
   max_stack_size ~$\in$~ positive integer:
       Stack CreateS(max_stack_size);
       Boolean IsFull(stack, max_stack_size);
       Stack Push(stack, item);
       Boolean IsEmpty(stack);
       Element Pop(stack); 
\end{codedef}
\end{frame}


\begin{frame}[t, fragile]
  \frametitle{Stack Abstract Data Type: Implementation}
  \begin{itemize}
  \item Using a one-dimensional array
    \begin{itemize}
    \item \texttt{stack[MAX\_STACK\_SIZE]}
    \item where \texttt{MAX\_STACK\_SIZE}: maximum number of entries
    \end{itemize}
  \end{itemize}
\begin{ncodedef}
#define MAX_STACK_SIZE  100
typedef struct {
    int key; // ~only key field~
             // ~can add/modify fields to meet the requirements~
} element;

element stack[MAX_STACK_SIZE];
int top = -1;
\end{ncodedef}
\end{frame}


\begin{frame}[t, fragile]
  \frametitle{Stack Abstract Data Type: implementation}
\texttt{IsEmpty(stack)}
  \begin{codedef}
    return(top < 0);
  \end{codedef}

\texttt{IsFull(stack);}

  \begin{codedef}
    return(top >= MAX_STACK_SIZE-1);
  \end{codedef}

\end{frame}


\begin{frame}[t, fragile]
  \frametitle{Stack Abstract Data Type: implementation}
\texttt{Push(stack, item)}

  \begin{codedef}
void push(int *ptop, element item){
    if (*ptop >= MAX_STACK_SIZE -1) {
       stack_full();
       return;
    }
    stack[++*ptop] = item;    
}
  \end{codedef}


\texttt{Pop(int *ptop);}

  \begin{codedef}
element pop(int *ptop){
    if(*ptop == -1)
        return stack_empty();
    return stack[(*ptop)--];
}
  \end{codedef}

\end{frame}


\begin{frame}[t]
  \frametitle{Stack Abstract Data Type: Application of Stack}
  \begin{itemize}
  \item Procedure calls/returns
  \item Syntactic analyzer
  \item converting recursive procedures to non-recursive procedures
  \end{itemize}
\end{frame}




\section{Queues} 
\begin{frame}[t]
  \frametitle{Queue Abstract Data Type: Characteristics}
  \begin{itemize}
  \item Ordered list
  \item All insertions are made at one end, called \texttt{rear}
  \item All deletions are made at the other end, called \texttt{front}
  \item which item is to be removed first?
    \begin{itemize}
    \item \texttt{FIFO} (First In First Out)
    \end{itemize}
  \item All items except \texttt{front}/\texttt{rear} items are hidden
  \end{itemize}
\end{frame}

\begin{frame}[t]
  \frametitle{Queue Abstract Data Type:  Insertion and Deletion}

Operation
{\footnotesize  \begin{centering}
\begin{tabular}{!{\onslide<2->}c<{\onslide<3->}c<{\onslide<4->}c<{\onslide<5->}c<{\onslide<6->}c}
insert A & $\rightarrow$ insert B &  $\rightarrow$ insert C &  $\rightarrow$ insert D &  $\rightarrow$ delete D  \\
    \end{tabular}
  \end{centering}}

{\onslide<1-> Queue state}

{\footnotesize  \begin{centering}
\begin{tabular}{!{\onslide<2->} c<{\onslide<2->}c<{\onslide<3->}c<{\onslide<3->}c<{\onslide<4->}c<{\onslide<4->}c<{\onslide<5->}c<{\onslide<5->}c<{\onslide<6->}c<{\onslide<6->}c}
       &                   &   &  &   &  & D & $\leftarrow$ rear &  &  \\
       &                   &   &  & C & $\leftarrow$ rear & C &  & D & $\leftarrow$ rear \\
       &                   & B & $\leftarrow$ rear & B &  & B &  & C &  \\
     A & $\leftarrow$ rear & A & $\leftarrow$ front & A & $\leftarrow$ front & A & $\leftarrow$ front & B & $\leftarrow$ front \\
       & $\leftarrow$ front & & &  & &  & &  & \\
    \end{tabular}
  \end{centering}}
\end{frame}

\begin{frame}[t, fragile]
  \frametitle{Queue Abstract Data Type: Implementation}
Simplest scheme
\begin{itemize}
\item one-dimensional array, and two variables: \texttt{front} and \texttt{rear}
\end{itemize}

\begin{ncodedef}
#define MAX_QUEUE_SIZE 100  
typedef struct {
    int key;
    /* other fields */
} element;

element queue[MAX_QUEUE_SIZE];
int rear = -1;
int front = -1;
\end{ncodedef}
\end{frame}

\begin{frame}[t, fragile]
  \frametitle{Queue Abstract Data Type: Implementation}
\texttt{IsEmptyQ(queue)}
\begin{codedef}
return (front == rear)
\end{codedef}
\texttt{IsFullQ(queue)}
\begin{codedef}
return rear == (MAX_QUEUE_SIZE-1)
\end{codedef}


\end{frame}

\begin{frame}[t, fragile]
  \frametitle{Queue Abstract Data Type}
\texttt{addq(*prear, element item)}
\begin{ncodedef}
void addq(int *prear, element item){
    if(*prear == MAX_QUEUE_SIZE - 1){
        queue_full();
        return;
    }
    queue[++*prear] = item;
}
\end{ncodedef}

\texttt{deleteq(*pfront, int rear)}
\begin{ncodedef}
element deleteq(int *pfront, int rear){
    if(*pfront == rear){ // rear ~is used to check for an empty queue~
        return queue_empty();
    }
    return queue[++*pfront];
}
\end{ncodedef}
\end{frame}

\begin{frame}[t]
  \frametitle{Queue Abstract Data Type: Example - Sequential Queue}
\textbf{Job Scheduling}: Creation of job queue
\begin{itemize}
\item in the OS which does not use priorities, jobs are processed in the order they enter the system
\end{itemize}
\begin{tabular}{c | c | c c c c | l}
  front & rear & Q[0] & Q[1] & Q[2] & Q[3] & Comments \\ \hline \hline
    -1  &  -1  &      &      &      &      & Queue is empty \\
    -1  &   0  &  J1  &      &      &      & Job 1 is added \\
    -1  &   1  &  J1  &  J2  &      &      & Job 2 is added \\
    -1  &   2  &  J1  &  J2  &  J3  &      & Job 3 is added \\
     0  &   2  &      &  J2  &  J3  &      & Job 1 is deleted \\
     1  &   2  &      &      &  J3  &      & Job 2 is deleted \\
\end{tabular}
\end{frame}

\begin{frame}[t, fragile]
  \frametitle{Queue Abstract Data Type: Example - Sequential Queue}
\textbf{Problem}
\begin{itemize}
\item Queue gradually shifts to the right
\item \texttt{queue\_full(rear == MAX\_QUEUE\_SIZE-1)} signal does not
  always mean that there are MAX\_QUEUE\_SIZE items in queue
\item There may be empty spaces available
\item data movement: O(MAX\_QUEUE\_SIZE)
\end{itemize}
\bigskip
\textbf{Solution:}
  \begin{itemize}
  \item<2-> Circular Queue
  \end{itemize}
\end{frame}


\section{Circular Queues} 
\begin{frame}[t]
  \frametitle{Circular Queues}
More efficient Queue representation
\begin{itemize}
\item  regard the array \texttt{queue[MAX\_QUEUE\_SIZE]} as circular
\item initially \texttt{front} and \texttt{rear} to \texttt{0} rather than \texttt{-1}
\item the \texttt{front} index always points one position counterclockwise from the first element in the queue
\item the \texttt{rear} index point to the current end of the queue
\end{itemize}
\end{frame}

\begin{frame}[t]
  \frametitle{Circular Queues}
empty and nonempty circular queues
\includegraphics[width=0.9\textwidth]{figures/fig04_cq.png}
\end{frame}

\begin{frame}[t]
  \frametitle{Circular Queues}
full circular queues
\includegraphics[width=0.9\textwidth]{figures/fig05_fcq.png}
\end{frame}

\begin{frame}[t, fragile]
  \frametitle{Circular Queues: Implementation}
  \begin{itemize}
  \item Use modulus operator for circular rotation
  \end{itemize}
\bigskip
\textbf{Circular rotation of the rear}
\begin{codedef}
rear = (rear + 1) % MAX_QUEUE_SIZE;
\end{codedef}

\textbf{Circular rotation of the front}
\begin{codedef}
front = (front + 1) % MAX_QUEUE_SIZE;
\end{codedef}
\end{frame}

\begin{frame}[t, fragile]
  \frametitle{Circular Queues: Implementation}
Add to a circular queue
\begin{itemize}
\item rotate rear before we place the item in the rear of the queue
\end{itemize}

\begin{ncodedef}
void addq(int front, int *prear, element item){
    *prear = (*prear + 1)  % MAX_QUEUE_SIZE;
    if (front == *prear) {
        queue_full(prear);
            /* reset rear and print error */
        return;
    }
    queue[*prear] = item;
}
\end{ncodedef}
\end{frame}

\begin{frame}[t, fragile]
  \frametitle{Circular Queues: Implementation}
Delete from a circular queue

\begin{ncodedef}
element deleteq(int *pfront, int rear){
    element item;
    if (*pfront == rear)
        return queue_empty();
        /* queue_empty returns an error key */
    *pfront = (*pfront + 1) % MAX_QUEUE_SIZE;
    return queue[*pfront];
}
\end{ncodedef}
\end{frame}

\begin{frame}[t]
  \frametitle{Circular Queues: Implementation notes}
Tests for a full queue and an empty queue are the same
\begin{itemize}
\item To distinguish between the case of full and empty, permit a maximum of \texttt{MAX\_QUEUE\_SIZE - 1}
\end{itemize}
\bigskip
No data movement necessary
\begin{itemize}
\item Ordinary queue: \texttt{O(n)}
\item Circular queue: \texttt{O(1)}
\end{itemize}

\end{frame}

\section{A Mazing Problem}
\begin{frame}[t]
  \frametitle{A Mazing Problem}
The representation of the maze
\begin{itemize}
\item two-dimensional array
\item element 0 : open path
\item element 1 : barriers
\end{itemize}
\includegraphics[width=0.7\textwidth]{figures/fig06_maze.png}
\end{frame}

\begin{frame}[t]
  \frametitle{A Mazing Problem: Allowable Movements}
\includegraphics[width=0.7\textwidth]{figures/fig07_moves.png}
\end{frame}

\begin{frame}[t]
  \frametitle{A Mazing Problem: Conditions}
\texttt{[row][col]} which is on border
\begin{itemize}
\item has only three (or two) neighbors
\item surround the maze by a border of 1's
\end{itemize}

\bigskip
\texttt{m*p} maze
\begin{itemize}
\item require $(m+2) *(p+2)$ array
\item entrance position: \texttt{[1][1]}
\item exit position: \texttt{[m][p]}
\end{itemize}
\end{frame}

\begin{frame}[t, fragile]
  \frametitle{A Mazing Problem: Data Type}
\begin{ncodedef}
typedef struct {
    short int vert;
    short int horiz;
} offsets

offsets move[8]; /* array of moves for each direction */    
\end{ncodedef}

\begin{tabular}{c | c | c | c}
  name & dir & \texttt{move[dir].vert} & \texttt{move[dir].horiz} \\ \hline
N  &  0 & -1 &  0 \\
NE &  1 & -1 &  1 \\
E  &  2 &  0 &  1 \\
SE &  3 &  1 &  1 \\
S  &  4 &  1 &  0 \\
SW &  5 &  1 &  -1 \\
W  &  6 &  0 &  -1 \\
NW &  7 & -1 &  -1 \\
\end{tabular}
\end{frame}

\begin{frame}[t, fragile]
  \frametitle{A Mazing Problem: Positioning of moves}
Position of next move
\begin{itemize}
\item move from current position: \texttt{maze[row][col]} to the next position \texttt{maze[next\_row][next\_col]}
\end{itemize}
\begin{ncodedef}
next_row = row + move[dir].vert;
next_col = col + move[dir].horiz;
\end{ncodedef}
\end{frame}

\begin{frame}[t]
  \frametitle{A Mazing Problem: Approach}
Maintain a second two-dimensional array, \texttt{mark}
\begin{itemize}
\item avoid returning to a previously tried path
\item initially, all entries are \texttt{0}
\item mark to 1 when the position is visited
\end{itemize}
\end{frame}

\begin{frame}[t, fragile]
  \frametitle{A Mazing Problem: Initial maze algorithm}
  \begin{ncodedef}
initialize a stack to the maze's entrance coordinates 
    and direction to north; 
while (stack is not empty) {
    /* move to position at top of stack */
    <row,col,dir> = delete from top of the stack;
    while (there are more moves from current position) { 
        <next_row, next_col> = coordinates of next move;
        dir = direction of move;
        if ((next_row == EXIT_ROW) && 
            (next_col == EXIT_COL)) 
            success;
        if (maze[next_row][next_col] == 0 && 
            mark[next_row][next_col] == 0) {
            mark[next_row][next_col] = 1;
            add <row, col, dir> to the top of the stack; 
            row = next_row;
            col = next_col;
            dir = north;
        } 
    }
}
printf("no path found\n");    
  \end{ncodedef}
\end{frame}

\begin{frame}[t, fragile]
  \frametitle{A Mazing Problem: Data Type}
\begin{ncodedef}
#define MAX_STACK_SIZE 100
typedef struct {
    short int row;
    short int col;
    short int dir;
} element;
element stack[MAX_STACK_SIZE];  
\end{ncodedef}

bound for the stack size
\begin{itemize}
\item the stack need only as many positions as there are zeroes in the maze
\end{itemize}
\end{frame}

\section{Evaluation of Expressions} 
\begin{frame}[t, fragile]
  \frametitle{Expressions}
  \begin{codedef}
    ~$x = a / b - c + d * e - a * c$~
  \end{codedef}
\bigskip

To understand the meaning of a expressions and statements
\begin{itemize}
\item figure out the order in which the operations are performed
\end{itemize}

Operator precedence hierarchy
\begin{itemize}
\item<determine the order to evaluate operators> 
\end{itemize}

associativity
\begin{itemize}
\item how to evaluate operators with the same precedence
\end{itemize}
\end{frame}


\begin{frame}[t]
  \frametitle{Precedence hierarchy for C}
{\footnotesize
  \begin{center}
  \begin{tabular}{c | c | c}
token & precedence & associativity \\ \hline
() [] -> . & 17 & left-to-right    \\
-\,- ++ & 16 & left-to-right    \\
-\,- ++ ! $\sim$ - + \& * sizeof & 15 & right-to-left \\
(type)  & 14 & right-to-left \\
* / \%  & 13 & left-to-right \\
+ -  & 12 & left-to-right \\
<< >> & 11 & left-to-right \\
> >= < <= & 10 & left-to-right \\
== !=  & 9 & left-to-right \\
\& & 8 & left-to-right \\
\string^ & 7 & left-to-right \\
$|$ & 6 & left-to-right \\
\&\&  & 5 & left-to-right \\
|| & 4 & left-to-right \\
?: & 3 & right-to-left \\
= += -= /= *= \%= <<= >>= \&= \string^= |= & 2 & right-to-left \\
, & 1 & left-to-right \\
  \end{tabular}
\end{center}
}
\end{frame}

\begin{frame}[t, fragile]
  \frametitle{Evaluation of Expressions}
\textbf{Human Style}
\begin{enumerate}
\item assign priority to each operator
\item use parenthesis and evaluate inner-most ones\\
     $(((A*(b+c))+(d/e))-(a/(c*d)))$
\end{enumerate}
\bigskip
\textbf{Compiler Style} (in postfix form)
\begin{enumerate}
\item translation (infix to postfix)
\item evaluation (postfix)
\end{enumerate}
\begin{codedef}
  prefix form: (operator) operand operand
  infix form: operand (operator) operand
  postfix form: operand operand (operator)
\end{codedef}
\end{frame}

\begin{frame}[t]
  \frametitle{Prefix, Infix, and Postfix Notation}
{\footnotesize  \begin{center}
    \begin{tabular}{c | c | c}
      Prefix & Infix & Postfix \\ \hline
 $+ 2~ * 3~ 4$  & $2+3*4$ & $2~ 3~ 4~* +$ \\
 $+ * a~ b~ 5$  & $a*b+5$ & $a~ b~ * 5~ +$\\
 $* + 1~ 2~ 7$  & $(1+2)*7$ & $1~ 2~ + 7~ *$ \\
 $/ * a~ b~ c$  & $a*b/c$ & $a~ b~ * c~ /$ \\
 $* / a~ + - b~ c~ d~ * - e~ a~ c$ 
    & $((a/(b-c+d))*(e-a)*c$ 
    & $a~ b~ c~ - d~ + / e~ a~ - * c~ *$ \\
$+ - / a~ b~ c~ - * d~ e~ * a~ c$ 
    & $a/b-c+d*e-a*c$ 
    & $a~b~/c~-d~e~*+a~c~*-$
    \end{tabular}
  \end{center}}
Evaluation of postfix expression
\begin{itemize}
\item scan left-to-right
\item place the operands on a stack until an operator is found
\item perform operations
\end{itemize}
\end{frame}

\begin{frame}[t]
  \frametitle{Evaluating Postfix Expression}
  \begin{center}
    $6~ 2~ / 3~ - 4~ 2~ * +$
  \end{center}

  \begin{center}
    \begin{tabular}{c | c c c | c}
      token & [0] & [1] & [2] & top \\ \hline
      6 & 6 &   &   & 0 \\
\onslide<2->      2 & 6 & 2 &   & 1 \\
\onslide<3->      / & 6/2 & & & 0 \\
\onslide<4->      3 & 6/2 & 3 & & 1 \\
\onslide<5->      - & 6/2-3 & & & 0 \\
\onslide<6->      4 & 6/2-3 & 4 & & 1 \\
\onslide<7->      2 & 6/2-3 & 4 & 2 & 2 \\
\onslide<8->      * & 6/2-3 & 4*2 & & 1 \\
\onslide<9->      + & 6/2-3+4*2 & & & 0 \\
    \end{tabular}
  \end{center}

\end{frame}


\begin{frame}[t]
  \frametitle{Evaluating Postfix Expression}
\texttt{get\_token()}
\begin{itemize}
\item used to obtain tokens from the expression string
\end{itemize}

\texttt{eval()}
\begin{itemize}
\item if the token is operand, convert it to number and push to the stack
\item otherwise
  \begin{itemize}
  \item pop two operands from the stack
  \item perform the specified operation
  \item push the result back on the stack
  \end{itemize}
\end{itemize}
\end{frame}


\begin{frame}[t, fragile]
  \frametitle{Evaluating of Expression}
\begin{ncodedef}
#define MAX_STACK_SIZE 100 /* maximum stack size */
#define MAX_EXPR_SIZE 100  /* max size of expression */

typedef enum {lparen, rparen, 
              plus, minus, 
              times, divide, 
              mode, eos, operand
             } precedence;  

int stack[MAX_STACK_SIZE]; /* global stack */
char expr[MAX_EXPR_SIZE];  /* input string */
\end{ncodedef}

represent stack by a global array
\begin{itemize}
\item accessed only through top
\item assume only the binary operator $+, -, *, /,$ and $\%$
\item assume single digit integer
\end{itemize}
\end{frame}

\begin{frame}[t, fragile, allowframebreaks]
   \frametitle{Function to evaluate a postfix expression}
\begin{ncodedef}
int eval(){
    precedence token;
    char symbol;

    int op1, op2;

    int n = 0;
    int top = -1;
  
    token = get_token(&symbol, &n); 
  
    while (token != eos) {
        if (token == operand)
            push(&top, symbol-’0’); 
        else {
            op2 = pop(&top); 
            op1 = pop(&top); 
  
            switch (token) {
                case plus: push(&top, op1+op2); 
                           break; 
                case minus: push(&top, op1-op2);
                           break; 
                case times: push(&top, op1*op2);
                           break; 
                case divide: push(&top, op1/op2);
                           break; 
                case mod: push(&top, op1%op2);
            } 
         }
         token = get_token(&symbol, &n); 
     }
     return pop(&top); 
}  
\end{ncodedef}

\end{frame}

\begin{frame}[t, fragile, allowframebreaks]
  \frametitle{Function to get a token}
  \begin{ncodedef}
precedence get_token(char *psymbol, int *pn) {
    *psymbol = expr[(*pn)++]; 
    switch (*psymbol)
        case ‘(‘ : return lparen;
        case ‘)‘ : return rparen;
        case ‘+‘ : return plus;
        case ‘-‘ : return minus;
        case ‘*‘ : return times;
        case ‘/‘ : return divide;
        case ‘%‘ : return mod;
        case ‘ ‘ : return eos;
        default : return operand; /* no error checking */
    } 
}    
  \end{ncodedef}
\end{frame}

\begin{frame}[t]
  \frametitle{Complexity}
  \begin{itemize}
  \item time: O(n) where n: number of symbols in expression
  \item space: stack \texttt{expr[MAX\_EXPR\_SIZE]}
  \end{itemize}
\end{frame}


\begin{frame}[t]
  \frametitle{Infix to Postfix}
Algorithm for producing a postfix expression from an infix one
\begin{enumerate}
\item fully parenthesize the expression
\item move all binary operators so that they replace their corresponding right parentheses
\item delete all parentheses
\end{enumerate}

e.g. $a/b-c+d*e-a*c$
\begin{enumerate}
\item $((((a/b)-c)+(d*e))-(a*c))$
\item ab/c-de*+ac*-
\end{enumerate}
\textit{Requires two pass}
\end{frame}

\begin{frame}[t]
  \frametitle{Infix to Postfix}
Form a postfix in one pass
\begin{itemize}
\item order of operands is the same in infix and postfix
\item order of operators depends on precedence
\item we can use stack
\end{itemize}

Simple expression: $a+b*c$
\begin{itemize}
\item $a~ b~ c~ * +$
\end{itemize}
\end{frame}

\begin{frame}[t]
  \frametitle{Infix to Postfix}
  \begin{itemize}
  \item Output operator with higher precedence before those with lower precedence
  \end{itemize}

\bigskip
Translation of $a+b*c$ to postfix
  \begin{center}
    \begin{tabular}{c | c c c | c | c}
      token & [0] & [1] & [2] & top & output \\ \hline
      a &  &   &   & -1 & a \\
\onslide<2->      + & + &  &  & 0 & a \\
\onslide<3->      b & + & & & 0 & ab \\
\onslide<4->      * & + & * & & 1 & ab \\
\onslide<5->      c & + & * & & 1 & abc \\
\onslide<6->      eos &  &  & & -1 & abc*+\\
    \end{tabular}
  \end{center}
\end{frame}

\begin{frame}[t]
  \frametitle{Infix to Postfix: Parenthesized expression}
parentheses make the translation process more difficult
\begin{itemize}
\item equivalent postfix expression is parenthesis-free
\end{itemize}

expression $a*(b+c)*d$
\begin{itemize}
\item yield $a~b~c+*d*$ in postfix
\end{itemize}

right parenthesis
\begin{itemize}
\item pop operators from a stack until left parenthesis is reached
\end{itemize}

\end{frame}

\begin{frame}[t]
  \frametitle{Infix to Postfix: Parenthesized expression}

Translation of $a*(b+c)*d$ to postfix
  \begin{center}
    \begin{tabular}{c | c c c | c | c}
      token & [0] & [1] & [2] & top & output \\ \hline
                  a &   &   &   & -1 & a \\
\onslide<2->      * & * &   &   & 0 & a \\
\onslide<3->      ( & * & ( &   & 1 & a \\
\onslide<4->      b & * & ( &   & 1 & ab \\
\onslide<5->      + & * & ( & + & 2 & ab \\
\onslide<6->      c & * & ( & + & 2 & abc \\
\onslide<7->      ) & * &   &   & 0 & abc+\\
\onslide<8->      * & * &   &   & 0 & abc+*\\
\onslide<9->      d & * &   &   & 0 & abc+*d\\
\onslide<10->   eos & * &   &   & 0 & abc+*d*\\
    \end{tabular}
  \end{center}
\end{frame}

\begin{frame}[t]
  \frametitle{Infix to Postfix}
A precedence-based scheme for stacking and unstacking operators
\begin{center}
  \includegraphics[width=0.7\textwidth]{figures/fig07_postfix.png}
\end{center}
\texttt{isp[stack[top]] < icp[token]} : push

\texttt{isp[stack[top]] $\geq$ icp[token]} : pop and print

\end{frame}

\begin{frame}[t, fragile]
  \frametitle{Infix to Postfix}
Use two types of precedence (because of the `(' operator)
\begin{itemize}
\item in-stack precedence (isp)
\item incoming precedence (icp)
\end{itemize}

\begin{ncodedef}
precedence stack[MAX_STACK_SIZE];
/* isp and icp arrays 
   -- index is value of precedence 
      lparen, rparen, plus, minus, 
      times divide, mode, eos */

static int isp[] = {0, 19, 12, 12, 13, 13, 13, 0};
static int icp[] = {20, 19, 12, 12, 13, 13, 13, 0};
\end{ncodedef}
\end{frame}

\begin{frame}[t, fragile, allowframebreaks]
  \frametitle{Infix to Postfix: the function}
Function to convert from infix to postfix
\begin{ncodedef}
void postfix(void) {
    char symbol;
    precedence token;
    int n = 0;
    int top = 0;
    stack[0] = eos;
    
    for (token = get_token(&symbol, &n); 
        token != eos; 
        token = get_token(&symbol, &n)) {

        if (token == operand) 
            printf("% c", symbol);
        else if (token == rparen) { 
            while (stack[top] != lparen) 
                print_token(pop(&top));

            pop(&top);
        } else {
            while (isp[stack[top]] >= icp[token]) 
                print_token(pop(&top));

            push(&top, token); 
        }
    }
    while ((token = pop(&top)) != eos)
        print_token(token); 

    printf("\n");
}
\end{ncodedef}
\end{frame}


\begin{frame}[t]
  \frametitle{Infix to Postfix}
postfix
\begin{itemize}
\item no parenthesis is needed
\item no precedence is needed
\end{itemize}

complexity
\begin{itemize}
\item time: O(r) where r: number of symbols in expression
\item space: S(n) = n where n: number of operators
\end{itemize}
\end{frame}

\begin{comment}
\section{Multiple Stacks Queues} 
\begin{frame}[t]
  \frametitle{Multiple Stacks and Queues}
\textbf{Multiple Stacks}

We need \textit{n} stacks simultaneously
\begin{itemize}
\item maximum size of each stack is unpredictable
\item size of each stack is dynamically varying
\item efficient memory utilization for multiple stacks is difficult
\end{itemize}
\end{frame}


\begin{frame}[t]
  \frametitle{Multiple Stacks and Queues}
Sequential mappings of stacks into an array
\begin{itemize}
\item \texttt{memory[MEM\_SIZE]}
\bigskip
\textbf{case n = 2}

The first stack:
\begin{itemize}
\item bottom element: \texttt{memory[0]}
\item grow toward \texttt{memory[MEM\_SIZE-1]}
\end{itemize}

The second stack:
\begin{itemize}
\item bottom element: \texttt{memory[MEM\_SIZE-1}
\item grow toward \texttt{memory[0]}
\end{itemize}
\end{itemize}

\end{frame}

\begin{frame}[t]
  \frametitle{Multiple Stacks and Queues}
  \begin{center}
    \includegraphics[width=0.8\textwidth]{figures/fig08_mulstack.png}
  \end{center}

\texttt{top[0] = -1}: stack 1 is empty

\texttt{top[1] = m}: stack 2 is empty
\end{frame}

\begin{frame}[t, fragile]
  \frametitle{Multiple Stacks and Queues: case n $\geq$ 3}
\begin{itemize}
\item \textit{boundary}: point to the position immediately to the left of the bottom element
\item \textit{top}: point to the top element
\end{itemize}
\begin{ncodedef}
#define MEM_SIZE 100
#define MAX_STACKS 10

element memory[MEM_SIZE];

int top[MAX_STACKS];
int boundary[MAX_STACKS];
int n; /* number of stacks, n < MAX_STACKS */
\end{ncodedef}
\end{frame}

\begin{frame}[t, fragile]
  \frametitle{Multiple Stacks and Queues: case n $\geq$ 3}
divide the array into roughly equal segments
\begin{ncodedef}
top[0] = boundary[0] = -1;
for(i = 1; i < n; i++)
    top[i] = boundary[i] = (MEM_SIZE/n)*i -1;
boundary[n] = MEM_SIZE -1;
\end{ncodedef}

  \begin{center}
    \includegraphics[width=0.8\textwidth]{figures/fig09_boundary.png}
  \end{center}
initial configuration for \textit{n} stacks in \texttt{memory[m]}
\end{frame}

\begin{frame}[t]
  \frametitle{Multiple Stacks and Queues: case n $\geq$ 3}
initially

\texttt{boundary[i] = top[i] = $\left\lfloor m/n\right\rfloor$ * i -1}

$i^{th}$ stack is empty
\begin{itemize}
\item \texttt{top[i] == boundary[i]}
\end{itemize}

$i^{th}$ stack is full
\begin{itemize}
\item \texttt{top[i] == boundary[i+1]}
\end{itemize}

\end{frame}

\begin{frame}[t, fragile]
  \frametitle{Multiple Stacks and Queues: case n $\geq$ 3}
push an item to the $i^{th}$ stack
\begin{ncodedef}
void push(int i, element item) { 
    /* add an item to the i-th stack */
    if (top[i] == boundary[i+1]) 
        stack_full(i);
    memory[++top[i]] = item; 
}    
\end{ncodedef}
\bigskip
pop an item from the $i^{th}$ stack
\begin{ncodedef}
element pop(int i) {
    /* remove top element from the i-th stack */
    if (top[i] == boundary[i]) 
        return stack_empty(i);
    return memory[top[i]--]; 
}
\end{ncodedef}
\end{frame}

\begin{frame}[t]
  \frametitle{Multiple Stacks and Queues: case n $\geq$ 3}
Ex. \texttt{n = 5, m = 20, push(1, x)}

\begin{center}
  \includegraphics[width=0.8\textwidth]{figures/fig10_ex.png}
\end{center}
\bigskip
\texttt{b[0]=-1 b[1]=3 b[2]=7 b[3]=11 b[4]=15}

\texttt{t[0]=-1 t[1]=7 t[2]=11 t[3]=13 t[4]=16}
\end{frame}

\begin{frame}[t]
  \frametitle{Multiple Stacks and Queues: case n $\geq$ 3}
  \begin{enumerate}
  \item find the least \texttt{j} for \texttt{i < j < n} such that there is a free space between stacks \texttt{j} and \texttt{(j+1)}
    \begin{itemize}
    \item i.e.) \texttt{t[j] < b[j+1]}
    \item move stack \texttt{i+1, i+2, ···,j} one position to the
      right creating a space between stacks \texttt{i} and
      \texttt{(i+1)}
    \end{itemize}
  \item if there is no such a j, then look up the left direction
  \item data movement: O(m)
  \end{enumerate}
\end{frame}
\end{comment}
\end{document}
